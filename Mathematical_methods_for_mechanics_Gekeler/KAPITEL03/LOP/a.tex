\documentclass[12pt,a4paper]{article}
\input myvor_e
%\input myvorab
\parindent1em
%\includeonly{a1}
\begin{document}
\title{A Note on Duality\\ in Linear Programming}
\author{Tobias Baumgarten and Eckart W. Gekeler\\
Mathematisches Institut A, Universit�t Stuttgart\\
70550 Stuttgart, Germany}
%e-mail: gekeler at mathematik.uni-stuttgart.de\\
\date{}
\begin{titlepage}
\maketitle
\begin{abstract}
%{\bf A Note on Duality in Linear Programming.}
Let $y = [y_1,\ldots, y_m]$ and $y_{1:p} = [y_1,\ldots, y_p]$, etc.  .  It is
shown that the `tableaus' of the primal problem $\Max \{ax, \; B^{1:p}x =
c^{1:p}, \; B^{p+1:m}x \leq c^{p+1:m}\}$ and the dual problem $\Min\{yc, \; yB
= a, \; y_{p+1:m} \geq 0\}$ are identical schemes in the optimum in case of
uniqueness and nondegeneracy if superfluous
entries are cancelled in the tableau of the dual problem.
%
\par
%{\bf Zur Dualit�t in der Linearen Programmierung.}
%Es wird gezeigt, da� die ``Tableaus'' des primalen Problems $\Max\{ax, \;
%B^{1:p}x = c^{1:p}, \; B^{p+1:m}x \leq c^{p+1:m} \}$ und des dualen Problems
%$\Min\{yc , \; yB = a, \; y_{p+1:m} \geq 0\}$ im Optimum �bereinstimmen, wenn
%die L�sungen eindeutig und nichtdegeneriert sind und wenn im dualen Tableau
%�berfl�ssige Eintr�ge gestrichen werden.
%\par
{\bf Key Words.} Linear Programming, Duality, Simplex Method.
\par
%{\bf AMS subject classification.}  90C05, 49M35.
\end{abstract}
\end{titlepage}
\newpage \hfill\\
\thispagestyle{empty}
\setcounter{page}{0}
%
%\newpage
\vspace*{5cm}
Address:
\bc{T. Baumgarten, E.W. Gekeler\\
Mathematisches Institut A\\
Universit�t Stuttgart\\
70550 Stuttgart, Germany\\
e-mail: gekeler at mathematik.uni-stuttgart.de\\
}
\ec

Let $C \in \Bbb{R}^m{}_n$ denote a matrix with $m$ rows and $n$ columns, let
$\Bbb{R}^n := \Bbb{R}^n{}_1, \; \Bbb{R}_n := \Bbb{R}^1{}_n$, and let $a \in
\Bbb{R}_n, \; B \in \Bbb{R}^p{}_n, \; c \in \Bbb{R}^p, \; d \in \Bbb{R}^m$.
We consider the pair of linear problems
\[
\ba{.}{ll}
 (P) & \Max\{a x, \; Bx = c, \; Cx \leq d \},\\
 (D) & \Min\{yc + zd, \; yB + zC = a, \; z \geq 0\}
\ea{.}
\]
and compare the tableaus of both problems without using slack variables;
cf. e.g. \cite{BeRi}, \cite{GroTer}.
\par
An application of the Lemma of {\sc Farkas}, cf. \cite{Spe}, \cite{BeRi},
yields the following characterization of a solution of (P) and (D):

\begin{theorem} \label{1} (a)
$x^{\ast} \in \Bbb{R}^n$ is a solution of (P) iff there exists a tripel
$(x^*,y^*,z^*) \in \Bbb{R}^n \times \Bbb{R}_p \times \Bbb{R}_m)$ satisfying
%
\[
\ba{.}{lll}
(i) & Bx^{\ast} = c, \; Cx^* \leq d	& \mbox{primal feasibility},\\
(ii) & y^*B + z^*C = a, \; z^* \geq 0	& \mbox{dual feasibility},\\
(iii) & z^*(Cx^* - d) = 0		& \mbox{complementary slackness}.
\ea{.}
\]
(b) $(y^*,z^*) \in \Bbb{R}_p \times \Bbb{R}_m$ is a solution of (D) iff there
exists a quadrupel $(y^*,z^*,u^*,v^*) \in \Bbb{R}_p \times \Bbb{R}_m \times
\Bbb{R}^n \times \Bbb{R}^m$ satisfying
\[
\ba{.}{lll}
(i)   & z^*C + y^*B = a, \; y^* \geq 0	& \mbox{primal feasibility},\\
(ii)  & B u^* + c = 0, \; Cu^* + d = v^* \geq 0 & \mbox{dual
feasibility},\\
(iii) & z^*v^* = 0			& \mbox{complementary slackness}.
\ea{.}
\]
\end{theorem}
%
In the Problem (P), $y^*$ and $z^*$ are called the {\sc Lagrange} multipliers
of the solution $x^*$, and in the problem (D), $u^*$ and $v^*$ are called the
{\sc Lagrange} multipliers of the solution $(y^*, z^*)$.
Writing
\begin{equation} \label{2}
x^* = - u^*, \; d - Bx^* = v^*,
\end{equation}
we see that the conditions in (a) and in (b) do coincide and, moreover,
\begin{equation} \label{3}
ax^* = y^*Bx^* + z^*Bx^* = y^*c + z^*d.
\end{equation}
Therefore we have the following inference to Theorem \ref{1}:
%
\begin{theorem} \label{4}
The problem (P) has a solution $x^*$ iff the problem (D) has a solution
$(y^*,z^*)$ and then (\ref{3}) holds.
\end{theorem}
%
%
In the sequel we suppose that the rank condition is fulfilled:
\[
\mbox{rank} (B) = p  \; \mbox{and} \; \mbox{rank} \ba{[}{l}B\\C \ea{]} = n.
\]
Instead of (P) we now consider the problem
%
\begin{equation} \label{8}
\Max\{ax, \; Cx \overset{[p]}{\leq} d\}
\end{equation}
%
where ``$\, \overset{[p]}{\leq} \,$'' indicates that in $Cx \leq d$ always the
first $p$ inequalities are active, i.e., hold as equations.  Using the {\sc
Matlab}-convention we write $C^{\A} := C(\A,:)$ and $C_{\A} := C(:,\A)$ with
any suitable index set $\A$.  Let then $x$ be an extreme point of (P), let
$\A(x)$ be the index set of a basis of $x$,
\[
\A(x) = \{1, \ldots,p,\rho _1, \ldots, \rho _{n-p}\}, \;
\N(x) = \{1, \ldots, m\} \backslash
\A(x),
\]
and let
\begin{equation} \label{9}
C^{\A} = \ba{[}{l} c^{\rho _1} \\ \vdots \\ c^{\rho _{n-p}}\ea{]}, \;
\ba{[}{l}B \\ C^{\A} \ea{]}^{-1} =: A = [A_B, \;  A_C] = [[a_1, \ldots, a_p],
\; [a_{p+1},\ldots, a_n]],
\end{equation}
where row vectors are denoted with upper indices and column vectors with lower
indices.  The columns of $A$ are the edges of the feasible set in the extreme
point $x$ with direction pointing to $x$.  Supposing that the problem (P) is
solvable and omitting {\sc Bland}'s rule we choose with $d = [\delta ^1,
\ldots, \delta ^m]^T$
\[ \ba{.}{ll}
j &:= \Min \Arg \Min \{\phi (k) := aa_k, \; k = p, \ldots, n\},\\

i &\dis := \Min \Arg \Min \{\psi(k) := \frac{c^kx - \delta ^k}{c^ka_j}, \;
c^ka_j < 0, \; k \in \N(x)\}.
\ea{.}
\]
Then a better --- at least not worse --- extreme point $\wi{x}$ is found from
$x$ by $ \wi{x} = x - \psi(i)a_j $ which means the row vector $c^{\rho _j}$ is
removed from and $c^i$ is taken into the row basis of $x$ yielding the basis of
$\wi{x}$.  The tableau of the primal problem (P) has thus the following form in
the extreme point $x$:
%
\begin{equation} \label{10}
{\bf P}(x) = [p^k{}_l] :=
\ba{[}{ll} A & x\\C^{\N}A & r\\w & \zeta  \ea{]}
= \ba{[}{lll} A_B & A_C & x\\
		     C^{\N}A_B & C^{\N}A_C & r\\
		     y & z_{\A} & \zeta
	   \ea{]}, \quad \ba{[}{l}x = Ac\\r = B^{\N}x - c^{\N}\\
				  y = aA_B\\ z_{\A} = aA_C\\
				  \zeta  = aAc
			 \ea{]}.
\end{equation}
The tableau ${\bf Q}(\wi{x}) = [q^k{}_l]$ of the extreme point $\wi{x}$ is
obtained from ${\bf P}(x)$ by the well-known {\sc Gauss-Jordan} step
%
\beqn
\ba{.}{lll}
q^i{}_j = 1/p^i{}_j & \mbox{(pivot element)}\,,\; &
q^k{}_j = p^k{}_j/p^i{}_j \,,\;k \neq i  \quad \mbox{(pivot column)}\,,\\
q^i{}_l = - p^i{}_l/p^i{}_j\,,\: l \neq j
& \mbox{(pivot row)}\,,\; & q^k{}_l =  p^k{}_l - p^k_jp^i{}_l/p^i{}_j
\quad\mbox{(others)}.
\ea{.}
\eeqn
%
\par
Let us now turn to the dual problem (D) which is is written in row form here
for convenience.  This problem has $m + p$ variables and $m + n$ side
conditions which can be written in the form
%
\begin{equation} \label{11}
[y, z]\ba{[}{rr}  B & 0 \\C & - I_m \ea{]} \leq [a, 0] \in \Bbb{R}_{m+n}
\end{equation}
%
where the first $n$ conditions are always active.  The matrix of these side
conditions has the full rank $m + p$.  A row vector $(y, z)$ is extreme point
iff besides the $n$ active restrictions $yB + zC = a \in \Bbb{R}_n$ at least
further $m - n + p$ conditions $y^i \leq 0$ are active, i.e., the equality sign
holds there.  The gradients of these conditions are independent by (\ref{11}).
Let $(y , z)$ be an extreme point of (D) and let --- by historical reasons ---
%
\[ \ba{.}{lll}
 \N(z) &= \{i \in \{1, \ldots, m\}, \; z^i = 0\}, & |\N(z)| = m - n + p,\\
\A(z)  &= \{1, \ldots, m\} \backslash \N(z),	  & |\A(z)| = n - p,\\
z_A    &:= z(\A(y)), \; z_N := z(\N(z)).
\ea{.}
\]
In the present ``row problem'' the gradients of active side conditions are
column vectors and the matrix $\wi{C}_{\N}$ of the gradients of the active
conditions --- corresponding to the matrix $C^{\A}$ in the problem (P) --- has
the following form after a suitable row permutation $Q$
\[
Q\wi{C}_{\N} = \ba{[}{cc}  B & 0 \\C^{\A} & 0 \\ C^{\N} & - I_{m-n+p}\ea{]},
\; Q\wi{B}_{\A} = \ba{[}{c}0 \\-I_{n-p} \\ 0 \ea{]}.
\]
We write again briefly
\[
\ba{[}{l}B \\ C^{\A}\ea{]}^{-1} =: A = [A_B, A_C], \; A_B \in \Bbb{R}^n{}_p,
\; A_C \in \Bbb{R}^n{}_{n-p}.
\]
Then the matrix $\wi{A}$ of the edges in the extreme point
$(y, z)$ has here the form
\[
\wi{A} \equiv \ba{[}{l}\wi{a}^1\\\vdots\\ \wi{a}^{m+p}\ea{]}
:= [\wi{C}_{\N}]^{-1} =
\ba{[}{cc} A & 0\\C^{\N}A & -I_{m-n+p}\ea{]}Q^T.
\]
The first $n$ rows of $\wi{A}$ cannot be chosen for descent
directions because they leave the feasible set.
In the present minimum problem, the optimality condition thus has the form
%
\begin{equation} \label{12}
v^{\rho _k} := - \wi{a}^{n+k}\ba{[}{l}c \\ d\ea{]} \geq 0, \; k = 1, \ldots,
m-n+p, \end{equation}
%
i.e. $- v^{\rho _k} \leq 0, \; k = 1, \ldots, m-n+p,$ where
%
\begin{equation} \label{13}
\wi{a}^{n+k}
= [c^{\rho _k}A, \; - [\delta ^k{}_l]_{l=1}^{m-n+p}]Q^T \in \Bbb{R}_{m+p}.
\end{equation}
%
($\delta ^k{}_l$ {\sc Kronecker} symbol). Writing
\[
Q\ba{[}{l}c\\ d\ea{]} = \ba{[}{l}c\\d^{\A}\\d^{\N}\ea{]},
\; \ba{[}{l}c\\ d \ea{]} = Q^T\ba{[}{l}c\\d^{\A}\\d^{\N}\ea{]}
\]
we obtain from (\ref{12})
%
\begin{equation} \label{14}
v^{\N} := d^{\N} - C^{\N}A_Cd^{\A} - C^{\N}A_Bc \geq 0 \in \Bbb{R}^{m-n+p}.
\end{equation}
\par
If $(y, z)$ is not optimal then (\ref{14}) is violated and
%
\[
\mbox{a column $e_r$ of} \;\ba{[}{c}0\\ 0\\- I_{m-n+p}\ea{]} \;
\mbox{and a column $e_s$ of} \ba{[}{c} 0\\
- I_{n-p}\\ 0 \ea{]} \; \mbox{are to be exchanged}.
\]
In the matrix $Q\wi{C}_{\N}$ this corresponds to an exchange of
{\bf row} of $C^{\N}$ with a row of $C^{\A}$.
We choose  $e_r, \; r = \N(z)_j$ with
%
\begin{equation} \label{15}
j = \Max \Arg_k \Max \{\phi(k) := \wi{a}^{n+k} \ba{[}{l}c \\ d \ea{]}, \; k
\in \{1, \ldots, m - n + p\}\},
\end{equation}
%
i.e. $j = \Max \Arg_k \Max \{- v^{\rho _k}, \; k \in \{1, \ldots, m-n+p\}\}$.
The search direction
$\wi{a}^{n+j}$ yields for $(\wi{y}, \wi{z}) = (y, z) -
\tau \wi{a}^{n+j}$ and $\tau > 0$
\[
[(y,z) - \tau \wi{a}^{n+j}]\ba{[}{l}c \\ d\ea{]}
<  yc + zd. \]
\par
For the computation of the optimum step length $\tau $ we have to substitute
$(\wi{y}, \wi{z})$ into the inactive conditions being simple sign conditions in
here,
%
\begin{equation} \label{16}
z_{\sigma _k} \geq 0 \Longleftrightarrow - (y, z)
\, e_{p + \sigma _k} \leq 0, \; k = 1, \ldots, n - p,
\end{equation}
recalling that $\A(z) = \{\sigma _1, \ldots, \sigma _{n-p}\}$.
Substitution of
 $(\wi{y}, \wi{z}) = (y,z) - \tau \wi{a}^{n+j}$ into
(\ref{16}) yields with	(\ref{13}) the condition for feasibility
\[
- z_{\sigma _k} + \tau [c^{\rho _j}A]_{p+k} \leq 0, \; k = 1, \ldots ,n - p.
\]
Supposing that the problem is solvable and omitting {\sc Bland}'s rule we
choose
%
\begin{equation} \label{17}
\dis i = \Min \Arg \Min \{\psi(k) := \frac{z_{\sigma
_k}}
{[c^{\rho _j}A]_{p+k}}, \;
%
[c^{\rho _j}A]_{p+k} > 0, \;  k \in \{1, \ldots, n - p\}\}.
\end{equation}
%
Then $\tau ^* := \psi(i) \geq 0$ is the optimum step length.
\par
For the tableau of the present row problem $\Max\{\wi{x}\wi{a}, \;
\wi{x}\wi{C} \leq \wi{d}\}$ we have in complete analogy to (\ref{10})
%
\begin{equation} \label{18}
\wi{{\bf P}} :=
\ba{[}{ccc} \wi{A} & \wi{A}\wi{C}_{\A} & \wi{w}\\
\wi{x} & \wi{r} & \wi{\zeta} \ea{]}
\end{equation}
%
where $\wi{x}$ is now the actual extreme point and $\wi{w}$ contains the
relevant parts of the multipliers:
\[
\wi{x} = [y, \; z_{\A}, \; z_{\N}] \in \Bbb{R}_{m+p}, \;
\wi{w} = - \ba{[}{l}u \\v^{\N}\ea{]} \in \Bbb{R}^{m+p}, \;
\wi{\zeta} = zd + yc \in \Bbb{R}.
\]
Moreover, we have
\[
\wi{r} = \wi{x}\wi{C}_{\A} - \wi{d}_{\A} = \wi{x}\ba{[}{l}0\\-I_{n-p}\\ 0\ea{]}
= -z_{\A}
\]
and
%
\[
\wi{A}\wi{C}_{\A} =
\ba{[}{ccc} A_B & A_C & 0\\ C^{\N}A_B & C^{\N}A_C & - I_{m-n+p}
\ea{]}\ba{[}{c} 0 \\ - I_{n-p} \\ 0 \ea{]}
= \ba{[}{c}- A_C \\ - C^{\N}A_C \ea{]}.
\]
Therefore we obtain for the tableau of the dual problem
%
\[
\wi{{\bf P}}(y,z)
=
\ba{[}{ccccc}
A_B	  & A_C       & 0	    & -A_C	      &  - u \\
C^{\N}A_B & C^{\N}A_C & - I_{m-n+p} & -C^{\N}A_C      &  - v^{\N}\\
y	  & z_{\A}    & z_{\N}	    & -z_{\A}	      & \wi{\zeta}
\ea{]}.
\]
In this tableau the second block column appears in the fourth block column
again with negative sign therefore the fourth block column can be cancelled.
The third block row can be cancelled, too, because the index set $\A(z)$ has
always to be updated and $z_{\N}$ equals zero by definition of the index set
$\N(z)$.  Hence recalling (\ref{2}) we obtain the desired result namely
\[
{\bf P}(x^*) = \wi{\bf{P}}_{red}(y^*,z^*)
\]
if $x^*$ is a unique and nondegenerate solution of the problem (P). (Then
$(y^*,z^*)$ is a unique and nondegenerate solution of the problem (D).)
In particular, the matrix $A$ has the same dimension in
both problems. Usually the last column of $\wi{{\bf P}}(y,z)$ is multiplied by
$-1$ in most presentations but a sign change of a not--pivot column does not
affect the global {\sc Gauss-Jordan} step.
\par
If $x$ is a non--optimum extreme point of (P) then $(y,z)$ in the
tableau ${\bf P}(x)$ is not feasible for (D) and, vice versa, if $(y,z)$ is a
non--optimum extreme point of (D) then the point $x$ associated to
this problem by (\ref{2}) is not feasible for (P). Therefore, if e.g. a primal
problem is solved by the dual method, i.e. the method for solving
(D), then the solution is approximated from the unfeasible domain and hence an
approximation is an unfeasible point.

\begin{thebibliography}{\hspace{5mm}}
%
\bibitem[BeRi]{BeRi} Best, M.J., und Ritter, K.:{\sl Linear Programming.}
Englewood Cliffs, N.Y.: Prentice Hall 1985.
%
\bibitem[Bla]{Bla} Bland, R.G.: {\sl New finit pivot rules for simplex
method.} Math. Operation Research 2 (1977), 103-107.
%
%
\bibitem[Las]{Las} Lasdon, L.S.: {\sl Optimization Theory for Large Systems.}
London: Macmillan 1970.
%
\bibitem[Naz]{Naz} Nazareth, J.L.: {\sl Computer Solutions of Linear
Programs.} New York--Oxford: Oxford University Press 1987.
%
\bibitem[Pad]{Pad} Padberg, M.: {\sl Linear Optimization and Extensions.}
Berlin--Heidelberg--New York: Springer--Verlag 1995.
%
\bibitem[Schr]{Schr} Schrijver, A.: {\sl Theory of Linear and Integer
Programming.}
New York: Wiley 1986.
\bibitem[Spe]{Spe} Spellucci, P.: {\sl Numerische Verfahren der Nichtlinearen
Optimierung.} Basel--Boston--Berlin: Birkh�user 1992.
%
\end{thebibliography}

\end{document}
