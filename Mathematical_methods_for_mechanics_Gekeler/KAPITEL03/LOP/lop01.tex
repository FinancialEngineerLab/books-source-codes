\section{Preface}
\hfill
\parbox{7cm}
{
Es ist nicht mehr einfach, wenn man's doppelt nimmt.\\
\hspace*{\fill} {\footnotesize Schw�bisches Sprichwort}
}
\par
\vspace{5mm}
In this year 1997, there are fifty years since the Simplex Algorithm came into
the world; cf.\  e.g.\	\cite{Dan}.  A very large amount of books and research
papers have appeared in the meanwhile on this procedure and the reader may ask
himself whether it is opportune to add here a further one.  But the authors
believe that they can still supply some new and different aspects of this
method which are also of interest for the very beginner.
\par
We build up directly on the famous result of {\sc Farkas} on sign bounded
solutions of linear systems of equations. Having established the concept of
duality by this way we focus on solving the primal-dual pair of problems
%
\[
\ba{.}{lllll}
(P) & \max\{a x, & B^{1:p}x = c^{1:p}, & B^{p+1:m}x \leq c^{p+1:m}\},\\
(D) & \min\{yc,  & yB = a, & y_{p+1:m} \geq 0\}
\ea{.} \; p \in \Bbb{N}_0.
\]
So the Problem (D) generalizes the classical simplex problem in a simple but
substantial way.  The problem (P) is solved by the ``projection method''
suggested in the well-known book \cite{BeRi} of {\sc M.Best} and {\sc K.Ritter}
and the
problem (D) is solved by the classical simplex method.	We show first that both
problems have identical ``tableaus'' in a unique point of optimum solution.
Thus they differ only in the choice of pivot rules and in the choice of the
initial point being either feasible for (P) or feasible for (D).  If the pivot
element is fixed in some way then also the exchange of a basis elements is the
same operation
in both methods.  In all algorithms developed here, the solution
$x$ of the problem (P) and the solution $y$ of the problem (D) are computed
together in every single algorithm but one of both is always approximated from
the unfeasible domain.	Moreover, the treatment of multiple solutions and
degenerations as well as sensitivity analysis
gain from the strict primal-dual way of consideration.	No slack variables are
used in this context but only in the construction of feasible points
(phase 1) where one additional variable is necessary.
\par
The principle idea of this volume is to deal first with the primal problem (P)
because of its generality and the possibility of easy geometric interpretation.
This way is very helpful in the development of further methods for special side
conditions.  So we construct for instance an algorithm in revised form for
problems (P) with bounded variables which includes problems of the form (D).
This algorithm shall also serve for a model in further special problems.  The
proposition of treating the problem (P) for primal problem was made first by
\cite{BeRi} but also many other ideas can be found in this volume which
are used more or less explicitely in the sequel.
\par
Every numerical analyst knows that one can meet with many surprises in the
implementation of a numerical device.  But since the {\sc Matlab} environment
is available the era of books on Numerical Methods containing no program should
draw to a close.  In an Appendix we present some {\sc Matlab} programs for the
projection method and for the simplex method being generalized in the above
indicated way and we also solve the associated phase-1 problems.  Here it is
our intention to show the several steps of development and not only to give an
unique algorithm for all seasons.  Naturally, all methods in tableau form can
be written in revised form where only the edge matrix $A$ and the index
vectors are updated.  The
algorithm ``proj06.m'' for bounded variables however is already written in that
form such that it can be be modified more easily to solve problems with sparse
design matrix in the side conditions.
