\section{Convex Sets and Cones}
\begin{definition} \label{d1}
(a) Let $x, y \in \Bbb{R}^n$ then
\beqn
[x,y] := \{z = \lambda x + (1 - \lambda )y, \: 0 \leq \lambda  \leq 1\}
\eeqn
is the straight line connecting $x$ and $y$.\\
(b) A set $\C \subset \Bbb{R}^n$ is {\bf convex} if
\beqn
\forall \; x, \, y \in \C: \; [x,y] \subset \C.
\eeqn
(c) A set $\K \subset \Bbb{R}^n$ is a {\bf cone} (``with top in
the origin'') if
\beqn
\forall \; x \in \K, \; \forall \; \alpha  \geq 0: \; \alpha x \in \K.
\eeqn
(c) A cone $\K$ is {\bf pointed} if
\beqn
\K \cap - \K = \{0\} \quad (\Longleftrightarrow x \in \K \wedge -x \in \K
\Longrightarrow x = 0).
\eeqn
(d) A {\bf convex} cone $\K$ defines a {\bf pre-order} by
\beqn
y \geq x \Longleftrightarrow x \leq y :\Longleftrightarrow y - x \in \K
\eeqn
with the properties
%
\beqn \ba{.}{lll}
\forall \: x \in \Bbb{R}^n: & x \leq x &\mbox{(reflexivity)},\\
x \leq y \wedge y \leq z &\Longrightarrow y \leq z &
\mbox{(transitivity)}.\\ \ea{.} \eeqn
%
(e) A pointed convex cone $\K$ defines a {\bf partial order} with the
additional property
\beqn
x \leq y \wedge y \leq x \Longrightarrow y = x \quad \mbox{(symmetry)}.
\eeqn
(f) If a pre-order is given by a convex cone then this cone is called
{\bf positive cone} or order cone.\\
(g) For a given positive cone $\K$ the set
%
\beqn
\K' := \{y \in \Bbb{R}_n, \; \forall \; x \in \K: yx \geq 0 \}
\quad \mbox{(row vectors)}
\eeqn
%
is the {\bf dual (positive) cone} (in $\Bbb{R}_n$).
\end{definition}
%
%
A positive cone $\K$ is never open but often closed.  The dual cone $\K'$ is
always closed.  In the space $\Bbb{R}^n$ of coordinates always $\K \subset
(\K')'$ holds and if $\K$ is closed then $\K = (\K')'$.
%
\begin{example} \label{b3}
If $\K \subset \Bbb{R}^n$ is a positive cone and $A \in
\Bbb{R}^m{}_n$ then
\beqn
A\K := \{y \in \Bbb{R}^m, \; \exists \; x \in \K: \; y = Ax \}
\eeqn
is a convex cone.
\end{example}
%
%
\begin{lemma} \label{l1}
Let $\emptyset \neq \C \subset \Bbb{R}^n$ be convex and let $0 \notin \C$.
Then there exists an $y \in \Bbb{R}_n$ and an $\alpha > 0$ such that $y x >
\alpha > 0$ holds for all $x \in \C$, i.e., the hyperplane $H = \{x \in
\Bbb{R}^n, \; yx = \alpha \}$ separates $\C$ and $\{0\}$ strongly.
\end{lemma}
%
%
Proof. The set
%
\beqn
\U := \{x \in \Bbb{R}^n, \; \|x\| \leq \beta \} \cap \C
\eeqn
%
is not empty for sufficiently large $\beta $ hence the continuous
function
$f: x \mapsto \|x\|$ attains its minimum on $\U$ at a point
$v \neq 0$. Because $\C$ is convex we have
\beqn
\|v + \lambda (x - v)\|^2 \geq \|v\|^2, \quad \lambda  \in [0,1],
\eeqn
for all $x \in \C$, or
\beqn
\lambda ^2(x - v)^T(x - v) + 2\lambda v^T(x - v) \geq 0, \quad \lambda  \in
[0,1].
\eeqn
It follows for $\lambda \to 0$ that $v^T(x - v) \geq 0$ hence
\beqn
v^Tx \geq v^Tv > v^Tv/4 =: \alpha  > 0.
\eeqn
%
This inequality yields the assertion for $y = v^T$.
\par
%
\begin{lemma} \label{l2} (Separation Theorem)
Let $\K \subset \Bbb{R}^n$ be a positive closed cone and let
$b \notin \K$.  Then there exists an $y \in \K'$ with $y b <
0$ (hence always $\K' \neq \emptyset$).
\end{lemma}
%
%
Proof.  We have $b \neq 0$ because of $0 \in \K$.  As $\K$ is
convex and closed, there exists a $y \in \Bbb{R}_n$ with
%
\begin{equation} \label{e2}
yb < \alpha  < \inf_{x \in \K}yx
\end{equation}
%
by a slight modification of Lemma \ref{l1}, and $\alpha < 0$ because of $0 \in
\K$.  Let $u \in \K$ with $yu < 0$. Because of $\lambda u \in \K, \:  \lambda
\geq 0$, there exists a $w = \lambda u$ with $yw = \lambda yu < \alpha
$.  This is a contradiction to (\ref{e2}).  Thus $yu \geq 0$ for all $u \in
\K$ hence $y \in \K'$ by definition.
\par
%
%
As an inference we obtain the so-called cone corollary:
%
%
\begin{lemma} \label{l3} Let $\K \subset \Bbb{R}^n$ be a
closed positive cone then
%
\beqn
x \in \K \Longleftrightarrow \forall \; y \in \K': \; yx \geq 0.
\eeqn
\end{lemma}
%
%
Proof. The left side implies the right side by definition and the negation of
the left side implies the negation of the right side by Lemma \ref{l2}.
%
%
\begin{lemma} \label{l4}
(Cone Inclusion Theorem) Let $\K, \: \L \subset
\Bbb{R}^n$ be positive cones and let $\L$ be closed then
\beqn
\K \subset \L \Longleftrightarrow \L' \subset \K'.
\eeqn
\end{lemma}
%
%
Proof.  Obviously the left side implies the right side.  The negation of the
left side implies the existence of a $x \in \K$ with $x \notin \L$.  Then there
exists an $y \in \L'$ with $y \, x < 0$ hence $y \notin \K'$ by the Cone
Corollary. This yields the negation of the right side.
\par
%
%
The next result is a necessary and sufficient condition for the existence of
positive solutions of linear systems of equations.
%
%
\begin{lemma} \label{l5} ({\sc Farkas} 1902)
Let $A \in \Bbb{R}^m{}_n, \; b \in \Bbb{R}^m$ and let $\K \subset \Bbb{R}^n$ be
a positive closed cone. Then
\beqn
\fbox{$
(b \in A \,\K) \Longleftrightarrow (y \, A \in \K' \Longrightarrow y \, b \geq
0). $}
\eeqn
\end{lemma}
%
Proof. (a) If the left side is true then $b = Ax$ with
$x \in \K$. If now $yA \in \K'$ then
\beqn
0 \leq yAx = yb.
\eeqn
(b) Let the right side be true. We have
\beqn
yA \in \K' \Longleftrightarrow \forall \: x \in \K: \; yAx \geq 0
\Longleftrightarrow y \in (A\K )' =: \Q '.
\eeqn
In the same way, with $\P := \{\alpha \, b, \; \alpha  \geq
0\}$,
\beqn
yb \geq 0 \Longleftrightarrow \forall \; \alpha  \geq 0: \; y(\alpha b)
\geq 0 \Longleftrightarrow y \in \P'.
\eeqn
With these notations, the right side says that $\Q ' \subset
\P '$ hence $\P \subset \Q $ by Lemma \ref{l4} thus $b \in A\K $.\\
%$\Box$ \hfill\\
%
%
By transposition of the result we obtain immediately
%
\begin{corollary} \label{f1}
Let $A \in \Bbb{R}^m{}_n, \; b \in \Bbb{R}_n$ and let
$\K \subset \Bbb{R}^m$ be a positive closed cone with dual cone
$\K'$. Then
\beqn
\fbox{$
(b \in \K'A) \Longleftrightarrow (Ax \in \K \Longrightarrow b \, x \geq
0). $}
\eeqn
\end{corollary}
%
Further, we define the following sets for a matrix $A \in \Bbb{R}^m{}_n$:
\beqn \ba{.}{lll}
\Range (A) &= \{y \in \Bbb{R}^m, \; \exists  \; x \in \Bbb{R}^n: \; y =
Ax\} & \mbox{{\bf range} of $A$}\\
\Ker (A) &= \{x \in \Bbb{R}^n, \; Ax = 0\} &
\mbox{{\bf kernel} or null space of $A$}.
\ea{.} \eeqn
%
Finally, if ${\cal U} \subset \Bbb{R}^n$ is a subspace then
\[
{\cal U}^{\perp} := \{x \in \Bbb{R}^n, \; \forall \; y \in {\cal U}: \; x^Ty =
0\}
\]
denotes the {\bf orthogonal complement} of ${\cal U}$ in $\Bbb{R}^n$.
%
\begin{theorem} \label{s1}
\[
\Range (A) = \Ker (A^T)^{\perp},
\]
\par
i.e., the range of a matrix is the orthogonal complement of the null space
of the transposed matrix.
\end{theorem}
%
Proof.
\beqn \ba{.}{l}
0 \neq y \in \Range (A)^{\perp} \Longleftrightarrow
\forall \; x \in \Bbb{R}^n: \; y^TAx = 0 \\
\Longleftrightarrow y^TA = 0 \Longleftrightarrow A^Ty = 0 \Longleftrightarrow
y \in \Ker (A^T).
\ea{.} \eeqn
%$\Box$ \hfill\\
%
As a special consequence we obtain
%
\begin{equation} \label{e3}
\Range (A) = \Bbb{R}^m \Longleftrightarrow \Ker (A^T) = \{0\}.
\end{equation}
In particular, if $A$ is quadratic then $\Ker (A^T) =
\{0\} \Longleftrightarrow \Ker (A) = \{0\}$ and (\ref{e3}) is\\[2mm]
%
{\bf Fredholm's Alternative}: {\it Either\\
(a) the system $Ax = y$ has a unique solution $y$
\\
or\\
(b) $Ax = 0$ has at least one nontrivial solution\\
but (a) and (b) does never hold.}
\par
Henceforth we stipulate that
\par
\[
\fbox{$
\K = \Bbb{R}^n_+,  \; \mbox{i.e.,} \; x \geq 0 \Longleftrightarrow
\; \forall \; i: \; x^i \geq 0,
$}
\]
\par
then $\K = \K'$ if $\Bbb{R}^n$ and $\Bbb{R}_n$ are identified canonically.
\par
For a matrix $A \in \Bbb{R}^m{}_n$, the set $\{y = Ax, \; x \geq 0\}$ is called
the positive cone spanned by the columns of $A$ and, in the same way, $\{y =
xA, \; x \geq 0\}$ is called the positive cone spanned by the rows of $A$.
\par
Theorem \ref{s1}, being fundamental for linear systems of equations, has an
analogon for sign-bounded solutions in the Lemma of
{\sc
Farkas}.  In order to emphasize the similarity and difference between both
results we present
Theorem \ref{s1} once more in the subsequent form (a) and the result of {\sc
Farkas} in the form (b)
:
%
\begin{corollary} \label{f2}
\beqn \ba{.}{llll}
(a) &Ax = y &\Longleftrightarrow  (zA = 0 &\Longrightarrow z \, y = 0),\\
(b) &Ax = y \wedge x \geq 0 &\Longleftrightarrow (zA \geq 0 &\Longrightarrow
z \, y \geq 0).
\ea{.} \eeqn
\end{corollary}
