\section{Simplex Method}
The projection method for the dual problem (\ref{e8}),
\[
\fbox{$
\min \{yc, \; yB = a, \; y_{\Q} \geq 0\}, \;
B \in \Bbb{R}^m{}_n, \; \rank (B^{\P}) = p, \;
\rank (B) = n,
$}
\]
is well-known under the name ``simplex method''.  We prefer the present row
form in order to use the {\bf same entries} as in the projection method without
transposition but note that the meaning of the vectors $a$ and $c$ has changed
here:  The column vector $c$ is now the gradient of the objective function and
the former gradient $a$ has become the right side of the side conditions.
Hoping that the reader accepts this less customary representation, we show in
this section that the simplex method leads to the same tableau (\ref{e36})
as in the primal problem after some cosmetic corrections.  By this conformance
we mean that the entries of the tableau do conform in dimensions and that the
tableaus coincide identically in a unique not-degenerate optimum.
\par
The problem (\ref{e8}) has $m$ variables, $y \in \Bbb{R}_m$, and $n + m- p$
side conditions of which the inequalities are simple sign conditions only.
Such simple bounds are omitted as far as possible in vertex methods.  We also
use the {\bf same} index vectors $\A (y)$ and $\N (y)$ as in the projection
method but now they are associated to a vertex $y$ of the present problem,
%
\begin{equation} \label{e19}
\ba{.}{rcl}
\A(y)  &:=& \{1, \ldots, p,\rho  _{p+1}, \ldots, \rho _n\},\\
\N(y)  &:=& \{\sigma  _1, \ldots, \sigma _{m - n}\}
= \I_m \backslash \A(y),
\ea{.}
\end{equation}
and we use also the same permutation matrix $P$ defined by
\[
[1, \ldots, m] = [\A (y), \N (y)]P.
\]
%
But, recalling that
\[
y_{\A \backslash \P} \geq 0, \; y_{\N} = 0,
\]
has been the optimality condition in the primal method we have to emphasize
that both index vectors now have a {\bf converse meaning} to that in the primal
problem:  Whereas in the primal problem the index set $\N(x)$ contains at least
all indices of inactive side conditions in a vertex $x$, i.e., $b^ix < \gamma
^i \Longrightarrow i \in \N(x)$, now $\A(y)$ contains at least all indices of
inactive side conditions in a vertex $y$, i.e., $y^i > 0 \Longrightarrow i \in
\A(y)$.
\par
%
The complete side conditions of the present row problem including equality
restrictions  have the form
\[
y\ba{[}{ccc} B^{\P} &        0           \\
                 B^{\Q} & -I_{m-p}       \\
\ea{]}  \leq [a,0].
\]
The gradients of the active conditions --- being column vectors here --- must
be independent in a vertex of the feasible domain hence the gradient
matrix of the present problem,
\[
\wi{B}_{\wi{\A}} := \ba{[}{cc} B^{\A}        &    O        \\
                          B^{\N}   &    - I_{m-n} \ea{]},
\]
must be regular in a vertex $y$.  This condition is fulfilled if and only if
the matrix $B^{\A} \in \Bbb{R}^n{}_n$ is regular for a suitable index vector
$\A(y)$.  Because of $\rank (B^{\P}) = p$ by assumption and (\ref{e19}), the
latter condition is fulfilled if and only if the rows
\[
b^{\rho _k}, \; k = p+1, \ldots, n
\]
of $B$ are linear independent. This requirement imples the following result
for the index set
\[
\fbox{$
\A^*(y) := \{i \in \I_m, \; i > p, \; y^i >0\}.
$}
\]
%
\begin{theorem} \label{s7} (Characterization of a vertex of the dual
problem) Let $\rank (B^{\P}) = p \; \mbox{and} \; \rank (B) = n$.
Then $y \in \Omega' = \{ y \in \Bbb{R}_m, \; yB = a, \;
y_{\Q} \geq 0\}$ is a vertex if and only if
\[
\forall \; i \in \A^*(y): \; b^i \; \mbox{linear independent}.
\]
\end{theorem}
%
Proof.  The first part of the assertion follows from the above stated fact that
the matrix $B^{\A}$ must have maximum rank in a vertex $y$.  Conversely,
if all $b^i, \; i \in \A^*(y)$, are linear independent then we can find an
index set $\A(y)$ such that $\rank (B^{\A \backslash \P}) = n - p$ if $\rank
(B) = n$.\\
%$\Box$ \hfill\\
%
Mutatis mutandis a vertex $y$ of $\Omega '$ is {\bf degenerate} if more than
$m$ side conditions are active in $y$, i.e., if the assertion of Theorem
\ref{s7} holds and $|\A^*(y)| < n - p$.  In particular, we have $p = 0$ in the
classical simplex problem
\[
\min \{yc, \; yB = a, \; y \geq 0\}.
\]
Then $y$ is a vertex of $\Omega '$ if and only if $|\A^*(y)| \leq n$ and all
$b^i, \; i \in \A^*(y)$, are linear independent.
\par
%
The gradient matrix $B^{\A}$ and the matrix $B^{\N}$ of the primal problem
have here analogues in the following matrices:
\begin{equation} \label{bab}
\wi{B}_{\wi{\A}} = P^T\ba{[}{cc}  B^{\A} & O \\ B^{\N} & - I_{m-n}\ea{]},
\; \wi{B}_{\wi{\N}} = P^T\ba{[}{c}O \\-I_{n-p} \\ O \ea{]},
\end{equation}
where $B^{\A}$ and $B^{\N}$ have the same meaning as in
the previous section, namely,
\[
B^{\A} = \ba{[}{c} b^{\rho _{1}}\\ \vdots \\ b^{\rho
_n}\ea{]},
\; B^{\N} = \ba{[}{c}b^{\sigma _1} \\ \vdots \\ b^{\sigma _{m-n}} \ea{]},
\; \rho _i = i, \; i = 1, \ldots, p.
\]
The row permutation with the matrix $P$ must be regarded here because the
vertex $y$ is also a row vector. We write again briefly
%
\[
A = [B^{\A}]^{-1} \in \Bbb{R}^n{}_n
\]
then the edge matrix $\wi{A}$ of the dual problem in a vertex $y$
reads
\begin{equation} \label{aaa1}
\wi{A} \equiv \ba{[}{l}\wi{a}^1\\\vdots\\ \wi{a}^m\ea{]}
:= [\wi{B}_{\wi{\A}}]^{-1} =
\ba{[}{cc} A & O\\B^{\N}A & -I_{m-n}\ea{]}P.
\end{equation}
Now we recall the dual feasibility condition (b) and the complement condition
(c) of Corollary \ref{f4} for the dual problem,
\begin{equation} \label{abc}
 B^{\P}u^* + c^{\P} = 0, \; B^{\Q}u^*  +c^{\Q} = v^* \geq 0,
\end{equation}
%
and, furthermore, we recall the relation (\ref{e11}) between a solution $x^*$
of the primal problem and the {\sc Lagrange} multiplier $z^* = (u^*,v^*)$ of
the dual problem,
\[
x^* = - u^*, \; r^* := Bx^* - c = - v^*.
\]
Because $B^{\A}x^* = c^{\A}$ hence $x^* = Ac^{\A}$, a substitution into
(\ref{abc}) yields the {\bf dual optimality  condition}
%
\[
v^{\N} = c^{\N} - B^{\N}Ac^{\A} \geq 0,
\]
or
%
\begin{equation} \label{e40}
\fbox{$
r^{\N} = - v^{\N} = B^{\N}Ac^{\A} - c^{\N} \leq 0.
$}
\end{equation}
\par
If $y$ is not optimum then condition (\ref{e40}) is violated and, following
the pattern of the primal method,  we have to remove a column of the
gradient matrix $P\wi{B}_{\wi{\A}}$ from the basis of $y$, namely
\[
\mbox{a {\bf column} $e_i$ of}  \; \ba{[}{c}O\\ O\\- I_{m-n}\ea{]}.
\]
Furthermore, we have to take a column of $P\wi{B}_{\wi{\N}}$ into the basis,
namely
\[
\mbox{a column $e_j$ of} \; \ba{[}{c} O\\ - I_{n-p}\\ O \ea{]}.
\]
But in the gradient matrix $P\wi{B}_{\wi{\A}}$, this exchange means that a row
$b^{\sigma _i}$ of $B^{\N}$ is taken into the row basis, i.e., the rows of
$B^{\A}$ and a row $b^{\rho _j}$ of $B^{\A}$ is removed form the basis into the
non-basis, i.e., the rows of $B^{\N}$.  We choose for $i$ that index for which
$-v^{\sigma _i}, \sigma _i \in \N (y)$ is maximum, i.e.,
\begin{equation}
\label{e42}
\fbox{$
i = \max \arg_k \max \{- v^{\sigma _k}, \; k \in \{ 1, \ldots, m - n\}\}.
$}
\end{equation}
%
Observe that $v^{\N} = \wi{A}^{(n+1):m}c$ hence $v^{\sigma _i} = \wi{a}^{n+i}c
< 0$ is minimum and the selected search direction $\wi{a}^{n+i}P^T$ yields a
proper descend for the new point $\wi{y} = y - \tau \wi{a}^{n+i}P^T$ in the
case where $\tau > 0$:
\[
[y - \tau \wi{a}^{n+i}P^T]c < yc.
\]
\par
The maximum feasible step width $\tau$ determines the vector $e_j$ which is
taken into the column basis of $y$ which corresponds to the remove of a row of
$B^{\A}$.  For this we subtitute again $\wi{y}$ into the inactive side
conditions in the same way as in the primal method and then enlarge $\tau$
until at least one of these conditions becomes ``active''.  However, the
inactive side conditions are purely sign conditions here and read
%
\[
\wi{y}_{\Q} = [y - \tau \wi{a}^{n+i}]_{\Q} \geq 0, \; \Q = \{p+1, \ldots, m\}.
\]
But, using the above introduced permutation matrix $P$,
\[
[y - \tau \wi{a}^{n+i}]P^T = [y_{\A}, y_{\N}] - \tau [b^{\sigma _i}A, -e_i]
= [y_{\A} - \tau b^{\sigma _i}A, \; y_{\N} + \tau e_i]
\]
where $e_i \in \Bbb{R}_{m-n}$ is the $i$-th row identity vector and $y_{\N} =
0$ by definition of the index set $\N (y)$. Therefore the conditions for
feasibility are
%
\begin{equation} \label{e44}
y_{\rho _k} - \tau b^{\sigma _i}a_k \geq 0, \; k = p+1, \ldots ,n.
\end{equation}
%
\par
{\bf Case 1.} $\forall \; k \in \{p+1, \ldots, n\}:  \; b^{\sigma _i}a_k \leq
0$.
Then (\ref{e44}) holds for all $\tau > 0$.  In this case the objective function
$yc$ is unbounded from below in the feasible domain and the problem has no
solution.
\par
{\bf Case 2.} $\exists \; k \in \{p+1, \ldots, n\}:  \; b^{\sigma _i}a_k >
0$. Then we choose
\begin{equation} \label{e45}
\fbox{$
\dis j = \min \arg_k \min \{\psi(k) := \frac{y_{\rho
_k}}
{b^{\sigma _i}a_k}, \;
%
b^{\sigma _i}a_k > 0, \;  k \in \{p+1, \ldots, n\}\}.
$}
\end{equation}
%
It follows that $\tau ^* := \psi(j) \geq 0$.  If the vertex is degenerate then
$y_{\A}$ contains zero elements which implies $\tau ^* = 0$ and can lead to
cycles. If however the vertex is not degenerate then $\tau ^* > 0$ and the old
vertex point is leaved.  We obtain by construction
\[
\wi{y}P^T = [y_{\A},0] -
\frac{y_{\rho_j}}{b^{\sigma _i}a_j}
[b^{\sigma _i}A, \; - e_i], \; e_i \in \Bbb{R}_{m-n},
\]
or, pointwise written, with fixed $i$ and $j$,
\beqn \ba{.}{llll}
\wi{y}_{\rho _j} &= 0, & \rho _j \in \A(y),\\
\wi{y}_{\sigma _i} &= \tau^* \geq 0, &   \sigma _i \in \N(y),\\
\wi{y}_{\rho _k} &\geq 0, & \rho _k \in \A(y) \backslash \{\rho _j\},\\
\wi{y}_{\sigma _k} &= 0, & \sigma _k \in \N(y) \backslash \{\sigma _i\}.\\
\ea{.}
\eeqn
Hence the new point $\wi{y}$ is a vertex point again if the row
vectors $b^1, \ldots, b^p$ and $b^l, \; l \in (\A(y) \backslash \{\rho _j\})
\cup \{\sigma _i\}$, are linear independent in the case where $\tau ^* > 0$.
Without loss of generality, let $p = 0$ and $j = n, \; b = b^{\sigma _i}, \;
[b[B^{\A}]^{-1}]_n \neq 0$ by (\ref{e45}).  Then the desired result is a
direct inference from the following Lemma
%
\begin{lemma} \label{s8}
Let
\[
   B = \ba{[}{l}b^1\\ \vdots \\ b^n \ea{]},
\; C = \ba{[}{l}b^1\\ \vdots \\ b^{n-1}\\b \ea{]}.
\]
If the matrix $B$ is regular and $[bB^{-1}]_n \neq 0$ then the matrix
$C$ is regular.
\end{lemma}
%
Proof. The matrix
\[
CB^{-1} = \ba{[}{ll} I_{n-1} & O\\x & [bB^{-1}]_n\ea{]}
\]
is regular by assumption hence both factors in the product on the left side
must be regular.\\
%$\Box$ \hfill \\
%
For the construction of the tableau, we write the present {\bf row problem} as
maximum problem, $\max\{\wi{x}\wi{a}, \; \wi{x}\wi{B} \leq \wi{c}\}$, then we
obtain in complete analogy to (\ref{e36})
%
\begin{equation} \label{e46}
\wi{{\bf P}}^*(x,y) :=
\ba{[}{ccc} \wi{A} & \wi{A}\wi{B}_{\wi{\N}} & \wi{w}\\
\wi{x} & \wi{r} &  \wi{f} \ea{]}.
\end{equation}
%
Here, $y = \wi{x}P$ is the actual vertex point and $\wi{w}$
contains the nontrivial part of the multipliers $(u,v)$ of the dual problem,
\[
y = [y_{\A}, \; y_{\N}]P \in \Bbb{R}_m, \;
\wi{w} = \ba{[}{l}u \\v^{\N}\ea{]} \in \Bbb{R}^m, \;
\wi{\zeta} = - yc \in \Bbb{R}.
\]
The vector $\wi{r}$ consists of the nontrivial part of the residuum of the
inequalities in $y$ again, cf.  (\ref{e36}) and (\ref{e8}), hence we have
here $\wi{r} = - y_{\A \backslash \P}$. Recall that
\[ \ba{.}{rclcrcl}
y_{\P} &=& [y_1, \ldots, y_p], &
y_{\A \backslash \P} &=& [y_{\rho _{p+1}}, \ldots , y_{\rho _n}],\\
A_{1:p} &=& [a_1, \ldots, a_p], & A_{p+1:n} &=& [a_{p+1}, \ldots a_n]
\ea{.}
\]
then
\[
\wi{A}\wi{B}_{\wi{\N}} =
\ba{[}{ccc} A_{1:p} & A_{p+1:n} & O\\ B^{\N}A_{1:p} & B^{\N}A_{p+1:n} &
- I_{m-n} \ea{]}\ba{[}{c} O \\ - I_{n-p} \\ O \ea{]}
%
= \ba{[}{c}- A_{p+1:n} \\ - B^{\N}A_{p+1:n} \ea{]}.
\]
Moreover, the last column of (\ref{e46}) can be multiplied with $-1$ because it
becomes never the pivot column.  Recalling (\ref{aaa1}), the complete tableau
of the simplex method reads in a vertex $y$
%
\begin{equation} \label{e47}
{\bf P}^*(x,y)
=
\ba{[}{ccccc}
A_{1:p}       & A_{p+1:n}       & O           & -A_{p+1:n}            &
-u \\ B^{\N}A_{1:p} & B^{\N}A_{p+1:n} & - I_{m-n} & -B^{\N}A_{p+1:n}
&    -v^{\N}\\
y_{\P}  & y_{\A \backslash \P}    & y_{\N}      & -y_{\A \backslash \P}
&    -\wi{f} \ea{]}.
\end{equation}
In this tableau, the second block column appears in the fourth column again
with negativ sign hence the fourth block column can be omitted entirely.  Also
the third block column can be deleted because the index vector $\A(y)$ must
be updated and because $y_{\N} = 0$ holds by definition of the index vector
$\N(y)$.  If we regard further once more the relation (\ref{e11}) between the
solution
of the primal problem and the {\sc Lagrange} multipliers $z = (u,v)$ of the
dual problem , namely
\[
u^* = - x^*, \;  v^* = d - Cx^* = - r^*, \; v^{\N*} = - r^{\N*},
\]
then it turns out that the tableau of the primal problem and the reduced
tableau of the dual problem coincide in the respective optimum,
%
\begin{equation} \label{e48}
{\bf P}(x^*,y^*)
=
\ba{[}{ccc}
A_{1:p}       & A_{p+1:n}                  &   x^* \\
B^{\N}A_{1:p} & B^{\N}A_{p+1:n}            &   r^{*\N}\\
y^*_{\P}      & y^*_{\A \backslash \P}     &   f^*
\ea{]},
\end{equation}
%
if the solution $x^*$ of the primal problem is not degenerate and unique.  (In
this case also the solution $y^*$ of the dual problem is not degenerate and
unique.) This fact allows the conclusion that the computational amount of work
in both problems is at least not very different in the mean.
\par
With the pivot rules (\ref{e42}) and (\ref{e45}), the element at the position
$(n+i,j)$ in the large tableau (\ref{e48}) is the pivot element,
\[
\fbox{$
[{\bf P}(x,y)]^{n+i}{}_j \; \mbox{pivot element}.
$}
\]
\par
If however $x$ is a non-optimum vertex of the primal problem then the {\sc
Lagrange} multipliers $y$ in the primal tableau is no feasible point of
the dual problem because the sign condition fails.  Conversely, if
$y$ is a non-optimum vertex of the dual problem then the point $x$ being
associated to the {\sc Lagrange} multiplier by means of (\ref{e11}) is no
feasible point of the primal problem.  In solving the primal problem in the
dual form, the solution is thus approximated from the unfeasible domain and an
approximation of the solution is necessarily an unfeasible point.  A
corresponding result holds for the dual problem.  In conclusion we thus may say
very coarsely that the dual method applies for problems containing sign
restrictions of the form $y_{\Q} \geq 0$.
