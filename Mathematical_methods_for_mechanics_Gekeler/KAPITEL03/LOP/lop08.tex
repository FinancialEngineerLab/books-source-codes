\section{Computation of a Vertex}
In general, the computation of a solution of a linear programming problem
divides into three steps:
\begin{itemize}
\item
computation of a feasible point,
\item
computation of a vertex starting from a feasible point,
\item
computation of an optimal vertex starting from a vertex.
\end{itemize}
The first step is called {\bf phase 1} and step 2 and 3 together are called
{\bf phase 2} of the problem.  The Vertex Theorem \ref{s5} describes how to
attain a vertex from a feasible point.  Suitable search directions $s$ are
found here by introduction of auxiliary side conditions which make the feasible
point of the problem to a vertex of an intermediate problem which then is
reduced to the original problem step by step.  The Main Theorem \ref{s2} then
yields a device for the computation of an optimal vertex point (solution) of
the feasible domain.
\par
%
{\bf (a) Primal Problem} \hfill\\
For the computation of a feasible point
\[
x \in \Omega = \{x \in \Bbb{R}^n, \; B^{\P}x = c^{\P}, \; B^{\Q}x \leq
c^{\Q}\}, \; c^{\P} \geq 0, \]
of the primal problem (\ref{e1}) we suppose without loss of
generality  $c^{\P} \geq 0$ and recall that $c = [\gamma ^i]_{i=1}^m$. We use
the notations
\[
b = \sum_{i=1}^pb^i, \; \delta  = \sum_{i=1}^p\gamma ^i, \;
e = [1, \ldots, 1]^T \in \Bbb{R}^{m-p}, \; \xi  \in \Bbb{R},
\]
and consider the slightly augmented linear problem
%
\[
\max \{- \xi + bx, \; B^{\P}x \leq c^{\P}, \; -e\xi + B^{\Q}x \leq
c^{\Q}, \; \xi  \geq 0\},
\]
%
i.e., the problem
%
\begin{equation} \label{e20}
\max \{ \wi{a}
\ba{[}{r}\xi \\ x \ea{]}, \; \wi{B}
\ba{[}{r}\xi \\ x\ea{]}
\leq \wi{c}\}
\end{equation}
with the entries
\[
\wi{a} = [-1, \; b], \;
\wi{B} = \ba{[}{rr} -1 & 0\\0 & B^{\P}\\-e & B^{\Q}\ea{]}, \;
\wi{c} = \ba{[}{l} 0\\ c^{\P} \\ c^{\Q} \ea{]}.
\]
%
For the problem
\[
\max \{ax, \; Bx \leq c\}
\]
the augmented auxiliary problem (\ref{e20}) reads
%
\begin{equation} \label{e21}
\max \{- \xi , \; Bx - e\xi  \leq c, \; \xi  \geq 0\}.
\end{equation}
Let
\[
\xi ^0 = \max\{0, -\gamma^{p+1}, \ldots, - \gamma ^m\}
\]
then $[\xi ^0, \; 0]$ is a feasible point of
the problem (\ref{e20}).  The objective function of the auxiliary problem is
bounded on the feasible domain of this problem because $\xi \geq 0$ and
\[
bx - \xi \leq bx \leq \delta .
\]
Hence the problem (\ref{e20}) has a solution. Also the rank condition is
fulfilled since
\[
\rank \ba{[}{rr} -1 & 0\\0 & B^{\P}\\-e & B^{\Q} \ea{]}
= \Rang \ba{[}{l}B^{\P}\\B^{\Q} \ea{]} + 1.
\]
%
%
\begin{lemma} \label{l6}
Let $(\xi ^*,x^*)$ be a solution of (\ref{e20}). Then $x^* \in
\Omega$ if and only if $bx^* - \xi ^* = \delta $.
\end{lemma}
%
Proof. Cf. \cite{BeRi}. Let $(\xi ^*, x^*)$ be a solution of
(\ref{e20}).\\
(a) Let $x^* \in \Omega$ then $(0,x^*)$ is a feasible point of
(\ref{e20}) and, in particular, $bx^* = \delta $. For $\xi ^* >
0$ we obtain
\[
\delta = bx^* - 0 > bx^* - \xi ^*
\]
in contradiction to the assumption that $(x^*,\xi ^*)$ optimum.  Hence $\xi ^*
= 0$.\\
%
(b) Let $bx^* - \xi ^* = \delta $.  Because of the feasibility of $(\xi
^*,x^*)$ with respect to (\ref{e20}), we have $B^{\P}x^* \leq c^{\P}$ and $\xi
^* \geq 0$.  Thus we have to show that $B^{\P}x^* \geq c^{\P}$ and $\xi ^* \leq
0$ then $x^*$ must be feasible and $\xi ^* = 0$.  However, if $b^ix^* < \gamma
^i$ for some $i \in \{1, \ldots, p\}$ or $\xi ^* > 0$ then
\[
bx^* - \xi ^* < \delta
\]
in contradiction to the assumption.
%$\Box$ \hfill
%
\begin{corollary} \label{f6} Let $(\xi ^*,x^*)$ be a solution of
(\ref{e20}). Then $x^* \in \Omega$ if and only if $\xi ^* = 0$.
\end{corollary}
\par
\par
Let us now put the question whether the final tableau of the auxiliary problem
(\ref{e20}) can be taken for start tableau in the original problem (\ref{e1}),
\[
\max \{ax ,\; B^{\P}x = c^{\P}, \; B^{\Q}x \leq c^{\Q}\}, \; c^{\P} \geq 0,
\]
in the case where the feasible domain is not empty.  The answer is positive as
the following result of \cite{BeRi} and \cite{Bau} shows.  Without loss of
generality we suppose that the index vector $\N(\xi ^0, 0)$ of the initial
feasible point $(\xi ^0, 0)$ of (\ref{e20}) is chosen such that
\[
\sigma _1 = 1 \in \N(\xi ^0, 0)
\]
and we also suppose that the pivot row in the tableau is chosen according to
the rule (\ref{e36a}),
\[
i = \dis \min \arg_k \min \{\frac{r^k}
{d^k{}_j}, \; d^k{}_j < 0, \; k \in \{1, \ldots, m - n \}\}.
\]
%
\begin{theorem} \label{tev1} (a)
Let $(\xi ^*, x^*)$ be a solution of (\ref{e20}) with basis index vector
$\A(\xi ^*,x^*)$ and let $\xi ^* = 0$ then $1 \in \A(\xi ^*,x^*)$.\\
(b)
Let $p = 0$, let $(\xi ^*, x^*)$ be a vertex of (\ref{e20}) with basis index
vector $\A(\xi ^*,x^*)$, let $\xi ^* = 0$ and $1 \in \A(\xi ^*,x^*)$
then $(\xi ^*,x^*)$ is a solution of (\ref{e20}).
\end{theorem}
%
Proof. (a) Suppose that $1 \notin \A(\xi ^*,x^*)$ then $1 \in \N(\xi ^*,x^*)$
and the solution is degenerate because (in the tableau of the auxiliary
problem)
\[
\wi{r}^1 = \wi{b}^1\ba{[}{c}\xi ^*\\x^*\ea{]} - \wi{\gamma }^1 = 0 - 0 .
\]
Without loss of generality we may suppose that in the step last but one
\[
\widehat{r}^1 = - \widehat{\xi }^0 < 0.
\]
Then $\wi{r}^1 = \widehat{r}^1 - \tau \widehat{d}^1{}_j < 0$ hence
also $\widehat{d}^1_j < 0$. Therefore we have
\[
\wi{\tau}  = \frac{\widehat{r}^1}{\widehat{d}^1{}_j}
\]
which yields $i = 1$ by the pivot rule (\ref{e36a}) in contradiction to the
assumption that $1 \notin \A(\xi ^*,x^*)$.\\
(b) If $1 \in \A(\xi ^*,x^*)$ and $\xi ^* = 0$ then in that vertex
\[
\wi{w}_{\A} = [-1, 0, \ldots,0]\wi{A} = [0, \ldots,0,1,0,\ldots,0] \geq 0
\]
hence $(\xi ^*,x^*)$ is optimum.\\
%$\Box$ \hfill\\
\par
By this result and Lemma \ref{f6} we have $\xi ^* = 0$ and $1 \in \A(\xi
^*,x^*)$ if $x^*$ is feasible for the original problem (\ref{e1}).  Therefore
we can cancel row $1$ and column $j$ with $1 = \rho _j \in \A(\xi ^*,x^*)$ from
the tableau of the auxiliary problem,
\begin{equation} \label{vt4}
\wi{\bf P} = \ba{[}{cc} \wi{A} &  \wi{x}\\ \wi{D} & \wi{r}\\
\wi{w} & \wi{\zeta } \ea{]}.
\end{equation}
Naturally, we also cancel  $1$ from $\A(\xi ^*,x^*)$ to obtain a basis
index vector $\A(x^*)$ of the feasible point $x^*$. Because $(\xi ^*,x^*)$
is an optimum vertex of the auxiliary problem with $\xi ^* = 0$ and $x^*$ is
feasible for the original problem, $x^*$ is a vertex of the domain
\[
\wi{\Omega} = \{x \in \Bbb{R}^n, \; Bx \leq c\}.
\]
satisfying $B^{\P}x^* = c^{\P}$ but not necessarily $\{1,\ldots,p\} \subset
\A(x^*)$. Therefore we have to take all $b^i, \; i = 1,\ldots,p$, into the
basis of $x^*$ by an intermediate iteration in order to obtain a
a basis index vector of the form $\A(x^*) = \{1,\ldots,p,\rho
_{p+1},\ldots,\rho _n\}$.
\par
In any case, the last row of (\ref{vt4}) must be changed for the original
problem because the objective function has changed.  So we have the last row of
(\ref{vt4}) to replace by
\[
w = aA, \; \zeta  = ax^*.
\]
There is also a second possibility to find a vertex point of the primal problem
(\ref{e1}) starting from a feasible point $v$ which however needs an
intermediate blow up of the tableau.  We may attach some additional trivial
side conditions to (\ref{e1}) such that $v$ becomes a vertex point of the new
auxiliary problem.  If e.g.  the primal problem is
\[
\max \{ax, \; Bx \leq c\}
\]
and $v$ is a feasible point then we may add the side conditions $x \leq v$,
i.e., we consider the problem
%
\[
\max \{ax, \; Bx \leq c, Ix \leq v\}
\]
which certainly has $v$ for vertex point because $\rank (I) = n$ and $v$ is
feasible in the original problem. The spurious side conditions $e^ix \leq
v^i$ are then replaced one after the other by a genuine side condition $b^ix
\leq \gamma ^i$ in a suitable way.
\par
In the original primal problem (\ref{e1}),
\[
\max \{ ax, \; B^{\P}x = c^{\P}, \; B^{\Q}x \leq c^{\Q} \}
\]
we remember that the $(p,m)$-matrix $B^{\P}$ has rank $p$ by assumption. Thus
we have to complete the rows of $B^{\P}$ to a row basis of $\Bbb{R}^n$. Let
\[
Q \cdot R = [B^{\P}]^T
\]
be a QR-decomposition of $[B^{\P}]^T$ with $Q^TQ = I$ and upper triangular
matrix
\[
R = \ba{[}{l}S\\ O \ea{]}
\]
where $S$ is a regular upper triangular matrix of dimension $p$ by assumption.
Then
\[
B^{\P}\cdot Q = [S^T, O^T], \; Q = [q_1, \ldots, q_n],
\]
hence the independent columns $q_{p+1}, \ldots, q_n$ span the kernel of
$B^{\P}$ and thus are a supplement of the rows of $B^{\P}$ to a row basis.
Accordingly, if $v$ is a feasible point again then we have to add the side
conditions
\[
q^T_ix \leq q^T_iv, \; i = p+1, \ldots, n.
\]
\par
%------------------------------------------------------------------------
{\bf (b) Dual Problem} \hfill\\
In the dual problem (simplex problem)
\begin{equation} \label{e21a}
\min\{yc, \; yB = a, \; y \geq 0\}, \; a \geq 0,
\end{equation}
we also suppose without loss of generality that $a \geq 0$.  Then an initial
vertex point may be computed directly by solving an augmented problem of the
same form, namely
%
\begin{equation} \label{e21b}
\min \{ve, \; yB + v = a, \; y \geq 0, \; v \geq 0\}
\end{equation}
%
where $e = [1, \ldots , 1]^T \in \Bbb{R}^n$.  This problem has
\[
(y,v) = (0,a), \; \A(y,v) = \{m+1,m+n\},
\]
for vertex point being possibly degenerated and $ve$ is bounded from
below for $v \geq 0$.  Hence the auxiliary problem (\ref{e21b}) is always
solvable and it therefore has also a vertex $(y^*,v^*)$ for solution
with index vector $\A(y^*,v^*)$.\\
%
{\bf Case 1.} If $v^* \geq 0$ and $v^* \neq 0$ then the original problem
(\ref{e21a}) has no feasible point and thus is unsolvable.\\
%
{\bf Case 2.} If $v^* = 0$ then $y^*$ is an intial vertex point for
(\ref{e21a}).
\par
However, in the dual problem we also have to transfer an index vector $\A(y^*)$
into phase 2.  If $y^*$ is not degenerate then $\A(y^*) =  A(y^*,v^*)$ else let
$\B(y^*)$ be obtained from $\A(y^*,v^*)$ by cancelling all indices $i > m$.
Then $|\B(y^*)| < n$ and we have to fill up $\B(y^*)$ to an index vector of
size $n$ by arbitrary indices $i \in \I_m \backslash \B(y^*)$.
%-------------------------------------------------------------------------
\par
Let us now turn to the general dual problem
%
\begin{equation} \label{e21c}
\min\{yc, \; yB = a, \; y_{\Q} \geq 0\}
\end{equation}
with auxiliary problem
%
\begin{equation} \label{e21d}
\min \{ve, \; yB + v = a, \; y_{\Q} \geq 0, \; v \geq 0\}
\end{equation}
%
and $p > 0$.  Here
we have to find an initial vertex $(y,v)$ for (\ref{e21c}) with index
vector of the form
\[
\A(y,v) = \{1,\ldots,p,\rho _{p+1},\ldots,\rho _n\}
, \; \rho _i \in \{p+1, \ldots, m+n\},
\]
because the first $p$ components of $y$ are now unrestricted.
\par
In the present case the original rank condition,
\[
\rank (B^{1:p}) = p, \; \rank(B) = n,
\]
must be slightly strengthened to
\[
\rank (B^{1:p}{}_{1:p}) = p, \; \rank (B) = n.
\]
Let for the moment $0 \in \Bbb{R}_k$ be the zero vector of size $k$, and let
$u \in \Bbb{R}_p$ be the unique solution of
\[
uB^{1:p}{}_{1:p} = a_{1:p}.
\]
Then we consider the point $(\wi{y},\wi{v})$ where
\[
\wi{y} = [u, \; 0_{m-p}], \; \wi{v} = [0_p, \; v], \; v \in \Bbb{R}_{n-p}.
\]
The side condition $yB + v = a$ splits up into two equations
\[
uB^{1:p}{}_{1:p} = a_{1:p}, \;
uB^{1:p}{}_{p+1:n}S + vS = a_{p+1:n}S
\]
where the sign matrix $S$ is a diagonal matrix with diagonal elements $\pm 1$.
Now, choosing
\[
vS = a_{p+1:n}S - uB^{1:p}{}_{p+1:n}S
\]
and $S$ such that $vS \geq 0$, the point $(\wi{y}, \wi{v})$ is a vertex of the
problem (\ref{e21d}) where the equality restriction are replaced by
\[
yB_{1:p} + v_{1:p} = a_{1:p}, \; yB_{p+1:n}S + v = a_{p+1:n}S.
\]
An index vector of a basis of $(\wi{y}, \wi{v})$ is
\[
\A(y,v) = \{1, \ldots, p, m+p+1, \ldots , m+n\}.
\]
The remaining part of the phase 1 problem remains the same as above and also
the index vector $\A(y^*)$ being transferred to phase 2 is found in the same
way.
\par
The auxiliary problem (\ref{e21d}) has the drawback that the number of
variables is $n+m$ instead of $m$ and that the present rank condition needs
possibly a column permutation of the design matrix $B$.  However, let
\[
Q \cdot R = B^T, \; L = R^T,
\]
be a QR-decomposition of $B^T$. Then the equality side conditions of
(\ref{e21a}) can be written as
%
\begin{equation} \label{e21g}
yBQ = yL = aQ =: \wi{a}.
\end{equation}
%
The submatrix $L^{1:p}{}_{1:p} = [l^i{}_k]_{i,k=1}^p$ is regular by the rank
condition $\rank (B^{\P}) = p$ hence the system
\[
uL^{1:p}{}_{1:p} = [a_{1:p}, \; 0]Q
\]
has a unique solution $u$.  Observing (\ref{e21g}) we may also solve the {\bf
modified} problem
%
\begin{equation} \label{e21h}
\min \{yc, \; yLS = \wi{a}S, \; y_{\Q} \geq 0 \}
\end{equation}
%
with side conditions
\[
y\ba{[}{cc}[LS]^{1:p} & 0\\[0mm]
           [LS]^{p+1:m} & -I_{m-p} \ea{]} \leq [\wi{a}S, \; 0_{m-p}]
\]
and the first $n$ conditions being always active. $S$ denotes the sign matrix
\[
S = [s^{i}{}_k], \; s = \ba{\{}{ll} \mbox{sign}(\wi{a}_i), & i = k,\\
                                    0              & \mbox{else}
                        \ea{.}
\]
($\mbox{sign}(0) = 1$). This problem has the degenerate vertex $y = 0$ with
index vector
\[
\A(y) = \{1,\ldots, p, n+1, \ldots,n+m-p\}.
\]
If $A = [[LS]^{\A}]^{-1}$ in optimum then the solution $x^*$ of the primal
problem is obtained by
\[
x^* = QSAc^{\A} = [B^{\A}]^{-1}c^{\A}.
\]
However, in the modified problem (\ref{e21h}) a possible sparsity of the design
matrix is destroyed therefor we use the auxiliary problem (\ref{e20}) in the
next section for the computation of a feasible point in the simplex
problem.
\par
% --------------------------------------------------------------------------
{\bf (c) Bounded Variables} \hfill\\
In the primal problem (\ref{e1}) with bounded variables,
\begin{equation} \label{e22}
\max \{ax, \; B^{\P}x = c^{\P}, \; B^{\Q}x \leq c^{\Q}, \; l \leq x \leq u \}
\end{equation}
let
\[ \ba{.}{ll}
l = [l^i] \in \{\Bbb{R} \cup \{- \infty\}\}^n, &
u = [u^i] \in \{\Bbb{R} \cup \{\infty\}\}^n,\\
\R = \{i \in \{1,\ldots,n\}, \; l^i \neq -\infty\}, &
{\cal S} = \{k \in \{1,\ldots,n\}, \; u^k \neq \infty\}.
\ea{.}
\]
Further, let
\[ \ba{.}{l}
\widehat{x} = [\widehat{x}^i]_{i=1}^n, \;
\widehat{x}^i = \ba{\{}{ll}
l^i, & i \in \R,\\
u^i, & i \in {\cal S}, \; i \notin \R,\\
0,   & \mbox{else}, \ea{.}               \\
\widehat{\xi } = \max \{b^1 \widehat{x} - \gamma ^1,
                 \ldots, b^n\widehat{x} - \gamma ^n, 0\}
\ea{.}
\]
and let again $b = \sum_{i=1}^pb^i, \; \delta  = \sum_{i=1}^p\gamma ^i, \; e
= [1,\ldots,1]^T \in \Bbb{R}^n$. Then $(\widehat{\xi },\widehat{x})$ is a
feasible point of the auxiliary problem
%
\begin{equation} \label{bv8}
\max\{[-(p+1)\xi  + bx]\ba{[}{c}\xi \\ x\ea{]},
\;[-e, \; B]\ba{[}{c} \xi  \\ x \ea{]} \leq c, \;
\ba{[}{c}0 \\ l\ea{]} \leq \ba{[}{c}\xi  \\ x\ea{]} \leq \ba{[}{c} \infty \\
u\ea{]} \}
\end{equation}
%
of which the dimension is also augmented by one.
If $(\xi ^*,x^*)$ is a solution of (\ref{bv8}) then
\begin{equation} \label{bv9}
bx^* -(p+1)\xi  = \delta
\end{equation}
if and only if $\xi ^* = 0$, and a slight modification of Lemma \ref{uuu} shows
that the objective function of (\ref{bv8}) has the value $\delta $ in optimum,
i.e., that (\ref{bv9}) holds in optimum if and only if $x^*$ is feasible for
(\ref{e22}).
