\section{Phase 1--2 in Simplex Method, Column Form}
As preparation, we consider the particular primal problem
\begin{equation} \label{uuu}
\max \{-ax, \; Bx = c, \; [0, \; - I_{m-p}]x \leq 0\},
\; B \in \Bbb{R}^n{}_m, \; c \geq 0,
\end{equation}
with the rank condition
\[
\rank (B_{1:p}) = p, \; \rank (B) = n.
\]
The phase-1--problem (\ref{e20}) then reads
\begin{equation} \label{kvk}
\max \{[-1, \; b]\ba{[}{c}\xi \\x\ea{]}, \;
\ba{[}{cc} -1 & 0\\ 0 & B \\ -e & [0, \; - I_{m-p}]
\ea{]}\ba{[}{c}\xi \\x\ea{]} \leq \ba{[}{c}0 \\ c \\ d\ea{]}
\end{equation}
where
\[
b = \sum_{i=1}^nb^i, \; \delta  = \sum_{i=1}^n\gamma ^i, \; d = 0 \in
\Bbb{R}^{m-p}.
\]
Here we have
\[
\xi ^0 = \max\{0, - d^1, \ldots, - d^{m-p}\} = 0
\]
with the components $d^k = 0$ of the vektor $d$ and thus the point $(\xi ,x) =
(0,0) \in \Bbb{R}^{m+1}$ is feasible for (\ref{kvk}). If $(\xi ^*,x^*)$ is a
solution of (\ref{kvk}) then $x^*$ is feasible for (\ref{uuu}) if $\xi ^* = 0$
or if $bx^* = \delta $ by Corollary (\ref{f6}) and Lemma (\ref{l6}).
Therefore we may cancel the additional variable $\xi $ entirely and ask whether
a solution $x^*$ satisfies $bx^* = \delta $.  So we consider the following
problem
%
\begin{equation} \label{sv1}
\max \{bx, \; \wi{B}x \leq \wi{c} \}
\end{equation}
with
\begin{equation} \label{sv2}
\wi{B} =  \ba{[}{cc} B_{1:p} & B_{p+1:m}\\ 0 & -I_{m-p}\ea{]}, \;
\wi{c} =  \ba{[}{c} c \\ 0\ea{]} \}, \; c \geq 0.
\end{equation}
having $n+m-p$ side conditions and $m$ variables.  Regarding the particular
form of the design matrix, we have to split up the column and row index set in
the following way:
\begin{equation} \label{ph3}
\ba{.}{rcl}
\{1, \ldots,m\} &\simeq& \{\W, \; p + \L, \; p + \U\},\\
\W &=& \{1, \ldots, p\},\\
\L &=& \{i \in \{1, \ldots, m - p\}, \; x^{p+i} > 0 \}, \; |\L| =: n_l,\\
\U &=& \{i \in \{1, \ldots, m - p\}, \; x^{p+i} = 0\}, \; |\U| =: n_u,\\[2mm]
\{1, \ldots,n+m-p\} &\simeq& \{\C, \; \D, \; n + \L, \; n + \U\},\\
\C &=& \{i, \; 1 \leq i \leq n, \; b^ix = \gamma ^i, \;
|\C| =: n_c\},\\
 \D &=& \{i, \; 1 \leq i \leq n, \; i \notin \C \}, \; |\D| = n - n_c.
\ea{.}
\end{equation}
Let $Q$ be the permutation matrix defined by
\[
Qx = \ba{[}{c}x^{\W}\\x^{\L}\\ x^{\U} \ea{]}.
\]
If $x$ is a vertex of the problem (\ref{sv1}) then it must
have  an basis index vector of the form
\[
\A(x) = \C(x) \cup (n+\U(x)), \; n_c + n_u = m,
\]
and a regular gradient matrix,
%
\[ \wi{B}^{\A}Q^T = \ba{[}{ccc}
         B^{\C}{}_{\W} & B^{\C}{}_{\L} & B^{\C}{}_{\U}\\
         O             &          O    & - I_{n_u}
                  \ea{]}.
\]
The regularity is fulfilled if and only if $n_c = p + n_l$ and if
the matrix
\[
A^{-1} := [B^{\C}{}_{\W} \; B^{\C}{}_{\L}]
\]
is regular. Then the edge matrix of the present problem is
%
\[
\wi{A} = [\wi{B}^{\A}]^{-1} = Q^T\ba{[}{cc} A & AB^{\C}{}_{\U}\\ O & -
I_{n_u}\ea{]}
\]
and the matrix of the inactive side conditions is
 %
\[ \wi{B}^{\N} = \ba{[}{ccc}
B^{\D}{}_{\W} & B^{\D}{}_{\L} & B^{\D}{}_{\U}\\
O & - I_{n_l} & O
\ea{]}Q.
\]
We obtain
\[
\wi{B}^{\N}\wi{A} =
\ba{[}{cc}
[B^{\D}{}_{\W} \; B^{\D}{}_{\L}]A & [B^{\D}{}_{\W} \;
B^{\D}{}_{\L}]AB^{\C}{}_{\U} - B^{\D}{}_{\U}\\
- A^{(p+1):n_c} & - A^{(p+1):n_c}B^{\C}{}_{\U}
\ea{]}
\]
and the full tableau of the problem (\ref{sv1}),
%
\[
\wi{{\bf P}} = \ba{[}{cc} \wi{A} & \wi{x} \\
                          \wi{B}^{\N}\wi{A} & \wi{r}\\
                          \wi{w} & \wi{f}
               \ea{]},
\]
has the form
\[
\wi{{\bf P}} = \ba{[}{ccc}
 A & AB^{\C}{}_{\U} & \wi{x}^{1:n_c}\\
 O & -I_{n_u}       & \wi{x}^{(n_c+1):m}\\[0mm]
[B^{\D}{}_{\W} \; B^{\D}{}_{\L}] A & [B^{\D}{}_{\W} \;
B^{\D}{}_{\L}]AB^{\C}{}_{\U} - B^{\D}{}_{\U}  & \wi{r}^{1:(n-n_c)}\\
- A^{(p+1):n_c} & - A^{(p+1):n_c}B^{\C}{}_{\U}    &
\wi{r}^{(n-n_c+1):(n-p)}\\
\wi{w}_{1:n_c} & \wi{w}_{(n_c+1):m} & \wi{f}
\ea{]}
\]
where
\[
\wi{x} = Qx, \; \wi{r} = \wi{B}^{\N}\wi{x} - \wi{c}^{\N}, \; \wi{w} = b\wi{A},
\; \wi{f} = bx.
\]
No information is lost here if the rows $p+1, \ldots, n$ are cancelled,
and the reduced tableau has the form
%
\begin{equation} \label{svs7}
\wi{{\bf P}}_{\mbox{red}} = \ba{[}{ccc}
 A^{1:p} & A^{1:p}B^{\C}{}_{\U} & \wi{x}^{1:p}\\[0mm]
[B^{\D}{}_{\W}, \, B^{\D}{}_{\L}] A & [B^{\D}{}_{\W}, \,
B^{\D}{}_{\L}]AB^{\C}{}_{\U} - B^{\D}{}_{\U}  & \wi{r}^{1:(n-n_c)}\\
- A^{(p+1):n_c} & - A^{(p+1):n_c}B^{\C}{}_{\U}    & -x^{\L}\\
\wi{w}_{1:n_c} & \wi{w}_{(n_c+1):m} & \wi{f}
\ea{]}
\end{equation}
with $n+1$ rows and $m+1$ columns.  In an exchange step we have to change the
$j$-th row $\wi{B}^{\A}Q^T$ with the absolute index $\rho _j$ for the $i$-th
row of $\wi{B}^{\N}Q^T$ with the absolute index $\sigma _i$.  Here we have to
make a difference between the following two cases according to the shape of
$\wi{B}^{\N}$:
\par
{\bf Case 1.} $i \in \{1, \ldots, n - n_c\}$. Then
\[
\ba{.}{rcl}
\C &=& \{\rho _1, \ldots, \rho _{n_c},\sigma _i\},\\
\U &=& \{\rho _{n_c+1}, \ldots, \rho _{j-1},\rho _{j+1}, \ldots, \rho
_{n_c+n_u} \},\\
\D &=& \{\sigma _1, \ldots, \sigma _{i-1}, \sigma _{i+1}, \ldots,
\sigma _{n-n_c}\},\\
\L &=& \{\sigma _{n-n_c+1},\ldots ,\sigma _{n-n_c+n_l},\rho _j\},
\ea{.}
\]
Here rows must be changed as well in the gradient matrix $\wi{B}^{\A}$ as
in the matrix $\wi{B}^{\N}$. But change of rows in the gradient matrix
implies change of columns in the edge matrix. Therefore, after swapping, we
have to move the j-th column at position $n_c + 1$ and the (i+p)-th row at
position $n$.
\par
{\bf Case 2.} $i \in \{n-n_c+1, \ldots, n - n_c + n_l\}$. Then
\[
\ba{.}{rcl}
\U &=& \{\rho _{n_c+1}, \ldots, \rho _{j-1},\sigma _i,\rho _{j+1},
\ldots, \rho _{n_c+n_u}\},\\
\L &=& \{\sigma _1, \ldots, \sigma _{i-1}, \rho _j, \sigma _{i+1}, \ldots,
\sigma _{n-n_c}\}.
\ea{.}
\]
Here the tableau does not change but only the meaning of some entries.
\par
The problem (\ref{sv1}) has the feasible point $x = 0 \in \Bbb{R}^n$ with
\[
\ba{.}{rcl}
\W(x) &=& \{1, \ldots, p\},\\
\L(x) &=& \emptyset,\\
\U(x) &=& \{p+1, \ldots, m\}
\ea{.}
\]
So it remains to add some trivial side conditions such that $x = 0$ becomes a
vertex of the second auxiliary problem.  We add the conditions $[-I_p, 0]x \leq
0$ then $x= 0$ satisfies $-I_mx = 0$ and thus is a vertex of the augmented
auxiliary problem
%
\begin{equation} \label{sv4}
\max\{bx, \; \ba{[}{cc} -I_p & O\\ B_{1:p} & B_{(p+1):m} \\ O & -I_{m-p}
\ea{]}x \leq \ba{[}{c}0 \\c \\ 0\ea{]}\}
\end{equation}
with $n+m$ side conditions. The basis index vector of the vertex $x = 0$ of
(\ref{sv4}) is
\[
\A(x) = \{1, \ldots,p,n+p,\ldots,n+m\}.
\]
But in the algorithm we write
%
\[
\A(x) = \{0, \ldots,0,n+p,\ldots,n+m\}
\]
where the first $p$ zeros indicate the spurious side conditions which have to
be inactivated at first.  The reduced first tableau of the second auxiliary
problem has the form
\[
{\bf P} = \ba{[}{ccc} -I_p & 0 & x^{1:p}\\
      -B_{1:p} & -B_{p+1:m} & -c\\
       \wi{w}_{1:p} & \wi{w}_{p+1:m} & \wi{f}
       \ea{]}
\]
with $ x^{1:p} = 0, \; \wi{w} = -b$ and $\wi{f} = 0$.  The regular matrix
$-I_p$ plays here the role of the key matrix $A^{-1}$.  The last row becomes
never pivot row hence the entire tableau can be multplied by $-1$ with a
suitable modification of the pivot rule as in the original simplex problem.
The solution $x^{\star}$ of this problem is a feasible point and also a vertex
of the original problem (\ref{uuu}) if $bx^{\star} - \delta = 0$.  Then
$Bx^{\star} = c$ holds but it may happen that not all $b^i, i = 1,\ldots,n$ are
in the basis of $x^{\star}$.  Then the above Case 1 must be repeated until $n_c
= n$.  Afterwards the basis index vector of $x^{\star}$ has the desired form
\[
\A(x^{\star}) = \{\rho _1, \ldots, \rho _m\}
\]
where
\[
\rho _k \in \ba{\{}{cc} \{1, \ldots, n\}, & 1 \leq k \leq n,\\
                        \U(x^{\star}), & k > n.
            \ea{.}
\]
But note that not necessarily $\rho _k = k, \; k = 1, \ldots, n$.
\par
At the end of phase 1, we have $\D(x^{\star}) = \emptyset$ hence the reduced
tableau (\ref{svs7}) has the form
%
\begin{equation} \label{svs8}
\wi{{\bf P}}_{\mbox{red}} = \ba{[}{ccc}
 A^{1:p} & A^{1:p}B^{\C}{}_{\U} & \wi{x}^{1:p}\\[0mm]
- A^{(p+1):m} & - A^{(p+1):m}B^{\C}{}_{\U}    & -x^{\L}\\
\wi{w}_{1:n} & \wi{w}_{(n+1):m} & \wi{f}
\ea{]}.
\end{equation}
Hence this tableau can be used for start tableau in phase 2 if the sign is
changed in the second block row and if the last row is replaced by
\[
[w, \; f] = [aA, \; ax].
\]
