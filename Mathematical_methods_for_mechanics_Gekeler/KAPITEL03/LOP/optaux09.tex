\documentclass[12pt,a4paper,leqno]{article}
\input myvor_e
\topmargin -1.5cm
\newcommand{\kopf}
{
\begin{tabular}{@{}p{11cm}@{}p{4cm}@{}}
MATHEMATISCHES INSTITUT A & {\large UNIVERSIT\"AT} \\
Prof. Dr. E. Gekeler	  & {\large STUTTGART} \\[10mm]
\end{tabular}
%\newline
}
\begin{document}
%
%
%-- PROJEKTIONSVERFAHREN I ------------------------------------------
\thispagestyle{empty}
\kopf
\vspace{-8mm}
\par
\ul{\bf Projektionsverfahren I} f\"ur Max $\{c^Tx\,|\,Ax \leq
b\}$,\\
$A \; (m,n)$--Matrix vom Rang $n$ mit Zeilen $a^i$ und Spalten $a_k$, $m \geq
n$.\\
Der aktuelle Wert von $x$ wird laufend mitberechnet.
\par
\ul{\bf Gegeben:} Ausgangsecke $x$ mit
$I_B(x)$ Indexmenge einer Basis von $x$, d.h. $|I_B(x)| = n$, und
$I_N(x)$ Indexmenge der \"ubrigen Zeilenindices von $A$.
(Reihenfolge beliebig, aber fest).
\beqn
B := [a^i]_{i \in I_B(x)}, \; N:=[a^i]_{i \in I_N(x)},
\; b_N = [\beta ^i]_{i \in I_N(x)}.
\eeqn
%
{\bf Vorbereitung.} Berechne
\beqn
\ba{.}{llll}
C = [c_i]_{1 \leq i \leq n} & := B^{-1}, & D = [d_i]_{1 \leq i \leq n}
:=NB^{-1}, &\\[1mm]
r =  [r^i]_{1\leq i \leq m-n} & := Nx - b_N, & y^T =  c^TB^{-1}, & z = c^Tx.
\ea{.} \eeqn
%
\ul{\bf 1. Schritt:} (Welche Suchrichtung $c_{\nu}$?)\\
Wenn $y \geq 0$, dann STOP ($x$ optimal), andernfalls berechne
\beqn \ba{.}{ll}
\nu := \mbox{Min \, Arg \, Min} \{ y_k \} &\mbox{(schnell)}, \,
c_{\nu} \, \mbox{neue Suchrichtung},\\
\nu := \mbox{Min \, Arg} \{ y_k, \; y_k < 0\} & \mbox{(sicher,
Bland--Regel, nur bei deg.  Ecke)}.
\ea{.} \eeqn
%
\ul{\bf 2. Schritt:} (Welche neue Schrittweite $\tau$?)\\
Wenn $d_{\nu} \equiv Nc_{\nu} \geq 0$, dann STOP (es
existiert keine optimale L\"osung),\\
andernfalls berechne
\beqn
\tau = :\mbox{Min }\{r^k/d^k_{\nu} \, | \, d^k_{\nu} < 0\}, \; \mu^{\ast} =
\mbox{Min }\{ k\,|\,r^k/d^k_{\nu} = \tau \} \,, \; \mu = \mu ^{\ast} + n\,.
\eeqn
%
\ul{\bf 3. Schritt:} (Austauschschritt.)
\beqn
P = [p_{kl}] :=
\left[
\begin{array}{ll}
C & x\\
D & r\\
y^T & z\\
\end{array}
\right]\;,\quad
(\left.
\begin{array}{r}
I_B(x)\\
I_N(x)\\
\end{array}
\right| \; P) \quad\mbox{ist Tableau des Verfahrens}\,.
\eeqn
Berechne $Q = [q_{k \, l}]$ gem\"a\ss
\beqn
\ba{.}{ll}
q_{\mu \nu} = 1/p_{\mu \nu} \quad\mbox{(Pivotelement)}\,,\; &
q_{k\nu}    = p_{k\nu}/p_{\mu\nu} \,,\;k \neq \mu
  \quad\mbox{(Pivotspalte)}\,,\\
q_{\mu l} = - p_{\mu l}/p_{\mu \nu}\,,\: l \neq \nu
\:\mbox{(Pivotzeile)}\,,\; & q_{k \, l} =  p_{k\,l} - p_{k\nu}p_{\mu
  l}/p_{\mu \nu} \quad\mbox{(\"ubrige Elemente)}\,.
\ea{.} \eeqn
%
Setze
\beqn \ba{.}{lll}
\ba{[}{l} C\\ D \ea{]}
&=& [q_{kl}]_{1 \leq k \leq m\,,\,1 \leq l \leq n} \;,\; x =
[q_{k,n+1}]_{1 \leq k \leq n}\:,\; r = [q_{k, n+1}]_{n+1 \leq k \leq
  m}\,,\\[1mm]
y^T &=& [q_{m+1, l}]_{1 \leq l \leq m}\:,\; z =
q_{m+1\,,\,n+1}\:,\\
I_B(x) &=& (\rho_1 , \ldots , \rho_{\nu -
  1}\,,\,\sigma_{\mu ^{\ast}}\,,\,\rho_{\nu +1} , \ldots \rho_n)\:,\;\\
I_N(x) & = & (\sigma_{n+1} , \ldots ,
\sigma_{\mu ^{\ast}-1}\,,\,\rho_{\nu}\,,\,\sigma_{\mu ^{\ast}+1}, \ldots
, \sigma_m)\\
\ea{.}
\eeqn
und kehre zum ersten Schritt des Verfahrens zur\"uck.\\
Die Aktualisierung der Indexmengen $I_B(x)$ und $I_N(x)$ im Austauschschritt
ist f\"ur den Algorithmus nicht notwendig, $x$ steht in der richtigen
Reihenfolge, weil nur Zeilen vertauscht werden.
%
%
%
%
%-- PROJEKTIONSVERFAHREN II -----------------------------------------
\newpage
\thispagestyle{empty}
\kopf
\vspace{-8mm}
\par
\ul{\bf Projektionsverfahren II} f\"ur Max $\{c^Tx\,|\,Ax \leq
b\}$,\\
$A \; (m,n)$--Matrix vom Rang $n$ mit Zeilen $a^i$ und Spalten $a_k$, $m \geq
n$.\\
Der aktuelle Wert von $x$ wird erst am Ende mitberechnet, daf\"ur ist das
Tableau kleiner.
\par
\ul{\bf Gegeben:} Ausgangsecke $x$ mit
$I_B(x)$ Indexmenge einer Basis von $x$, d.h. $|I_B(x)| = n$, und
$I_N(x)$ Indexmenge der der \"ubrigen Zeilenindices von $A$.
(Reihenfolge beliebig, aber fest).
\beqn
B := [a^i]_{i \in I_B(x)}, \; N:=[a^i]_{i \in I_N(x)},
\; b_N = [\beta ^i]_{i \in I_N(x)}.
\eeqn
%
{\bf Vorbereitung.} Berechne
\beqn
\ba{.}{llll}
C = [c_i]_{1 \leq i \leq n} & := B^{-1}, & D = [d_i]_{1 \leq i \leq n}
:=NB^{-1}, &\\[1mm]
r =  [r^i]_{1\leq i \leq m-n} & := Nx - b_N, & y^T =  c^TB^{-1}, & z = c^Tx.
\ea{.} \eeqn
%
\ul{\bf 1. Schritt:} (Welche Suchrichtung $c_{\nu}$?)\\
Wenn $y \geq 0$, dann STOP ($x$ optimal), andernfalls berechne
\beqn \ba{.}{ll}
\nu := \mbox{Min \, Arg \, Min} \{ y_k \} &\mbox{(schnell)}, \,
c_{\nu} \, \mbox{neue Suchrichtung},\\
\nu := \mbox{Min \, Arg} \{ y_k, \; y_k < 0\} & \mbox{(sicher,
Bland--Regel, nur bei deg.  Ecke)}.
\ea{.} \eeqn
%
\ul{\bf 2. Schritt:} (Welche neue Schrittweite $\tau$?)\\
Wenn $d_{\nu} \equiv Nc_{\nu} \geq 0$, dann STOP (es
existiert keine optimale L\"osung),\\
andernfalls berechne
\beqn
\tau = :\mbox{Min }\{r^k/d^k_{\nu} \, | \, d^k_{\nu} < 0\}, \; \mu =
\mbox{Min }\{ k\,|\,r^k/d^k_{\nu} = \tau \} \, .
\eeqn
%
\ul{\bf 3. Schritt:} (Austauschschritt.)
\beqn
P = [p_{kl}] :=
\left[
\begin{array}{ll}
D & r\\
y^T & z\\
\end{array}
\right]\;,\quad
(\left.
\begin{array}{r}
I_B(x)\\
I_N(x)\\
\end{array}
\right| \; P) \quad\mbox{ist Tableau des Verfahrens}\,.
\eeqn
Berechne $Q = [q_{k \, l}]$ gem\"a\ss
\beqn
\ba{.}{ll}
q_{\mu \nu} = 1/p_{\mu \nu} \quad\mbox{(Pivotelement)}\,,\; &
q_{k\nu}    = p_{k\nu}/p_{\mu\nu} \,,\;k \neq \mu
  \quad\mbox{(Pivotspalte)}\,,\\
q_{\mu l} = - p_{\mu l}/p_{\mu \nu}\,,\: l \neq \nu
\:\mbox{(Pivotzeile)}\,,\; & q_{k \, l} =  p_{k\,l} - p_{k\nu}p_{\mu
  l}/p_{\mu \nu} \quad\mbox{(\"ubrige Elemente)}\,.
\ea{.} \eeqn
%
Setze
\beqn \ba{.}{lll}
D &=& [q_{kl}]_{1 \leq k \leq m - n \,,\,1 \leq l \leq n} \;,\;
r = [q_{k, n+1}]_{1 \leq k \leq m-n}\,,\\[1mm]
y^T &=& [q_{m-n+1, l}]_{1 \leq l \leq m}\:,\; z =
q_{m-n+1\,,\,n+1}\:,\\
I_B(x) &=& (\rho_1 , \ldots , \rho_{\nu -
  1}\,,\,\sigma_{\mu }\,,\,\rho_{\nu +1} , \ldots \rho_n)\:,\;\\
I_N(x) & = & (\sigma_{n+1} , \ldots ,
\sigma_{\mu -1}\,,\,\rho_{\nu}\,,\,\sigma_{\mu +1}, \ldots
, \sigma_m)\\
\ea{.}
\eeqn
und kehre zum ersten Schritt des Verfahrens zur\"uck. $z$ ist
optimaler Wert der Zielfunktion. Berechnung von x:
\beqn
B := [a^i]_{i \in I_B(x)}, \; b_B = [\beta ^i]_{i \in I_B(x)}, \; x =
B^{-1}b_B.
\eeqn
%
%
%
%-- PROJEKTIONSVERFAHREN III (MIT ZULAESSIGEM PUNKT) ----------------
\newpage
%\setcounter{page}{1}
\thispagestyle{empty}
\kopf
\vspace{-5mm}
\par
\ul{\bf Projektionsverfahren} f\"ur $\mbox{Max }\{c^T x \,|\,Ax \leq
b\}$\,\\
$A$ (m,n)--Matrix vom Rang n mit Zeilen $a^i$ und Spalten $a_k, \; m \geq n$.\\
\ul{\bf Gegeben:} \ul{Zul\"assiger} Punkt $x \in \R^n$ mit $Ax \leq b$ und
\beqn
I_B(x) = [\rho _1,\ldots, \rho _n] =  [0,\ldots,0],\quad
I_N(x) = [1,\ldots,m],
\eeqn
\beqn \ba{.}{lll}
B := I = B^{-1} = C = [c_i]_{1\leq i\leq n}, &D:= NB^{-1} = N := A, &\\[0,7ex]
r = Nx - b, & y^T = c^TB^{-1}, & z = c^Tx.\\
\ea{.} \eeqn
%
\ul{\bf Schritt 1.1:} (Berechne Suchrichtung $s$.)\\
Wenn $I_B(x) > 0$, dann gehe zu Schritt 1.2.\\
Berechne
\beqn
\nu  := \mbox{Min Arg Max}\{|y_i|, \; I_B(x)_i = 0\}.
\eeqn
Wenn $y_{\nu} = 0$, dann gehe zu Schritt 1.2.\\
Wenn $y_{\nu} > 0$, dann setze $s = - c_{\nu }$, \quad
Wenn $y_{\nu} < 0$, dann setze $s = c_{\nu }$.\\
Gehe zu Schritt 2.\\
\ul{\bf Schritt 1.2:}\\
Wenn $y \geq 0$ und $I_B(x) > 0$, dann STOP (x optimal), andernfalls berechne
\beqn \ba{.}{ll}
\nu := \mbox{Min \, Arg \, Min} \{ y_k, \; I_B(x)_k > 0 \}
&\mbox{(schnell)},\\
\nu := \mbox{Min \, Arg} \{ y_k, \; y_k < 0, \; I_B(x)_k > 0 \} &
\mbox{(sicher, Bland--Regel, nur bei deg.  Ecke)}.\\
\ea{.} \eeqn
Setze $s = c_{\nu }$.\\
%
\ul{\bf Schritt 2:} (Welche neue Schrittweite $\tau$?)\\
Wenn $d_{\nu} \equiv Ns \geq 0$, dann STOP (es
existiert keine optimale L\"osung),\\
andernfalls berechne
\beqn
\tau = :\mbox{Min }\{r^k/d^k_{\nu} \, | \, d^k_{\nu} < 0\}, \; \mu^{\ast} =
\mbox{Min }\{ k\,|\,r^k/d^k_{\nu} = \tau \} \,, \; \mu = \mu ^{\ast} + n\,.
\eeqn
%
\ul{\bf Schritt 3:} (Austauschschritt.)\\
Wie beim Projektionsverfahren.
Setze anschlie\ss end
\beqn \ba{.}{l}
I_B(x) = (\rho_1 , \ldots , \rho_{\nu -
  1}\,,\,\sigma_{\mu ^{\ast}}\,,\,\rho_{\nu +1} , \ldots \rho_n)\:,\;\\
I_N(x) = (\sigma_1 , \ldots ,
\sigma_{\mu ^{\ast}-1}\,,\,\rho_{\nu}\,,\,\sigma_{\mu ^{\ast}+1}, \ldots
, \sigma_m).
\ea{.}
\eeqn
Wenn $\rho _{\nu } = 0$ dann setze
\beqn
I_N(x) = (\sigma_{n+1} , \ldots ,
\sigma_{\mu ^{\ast}-1}\,,\,\sigma_{\mu ^{\ast}+1}, \ldots
, \sigma_m)
\eeqn
und streiche die Zeile mit der Nummer $\sigma _{\mu ^{\ast}}$ in $r = Ax -
b$ und in $D$.\\
Kehre zum ersten Schritt des Verfahrens zur\"uck.\\
Zur eventuellen Berechnung eines $x$ mit $Ax \leq b$ betrachte man
das Problem
\beqn
(^{\ast}) \quad \mbox{Min } \{\lambda \,|\, Ax - \lambda e \leq
b\,,\: \lambda \geq 0\}, \quad (x\,,\,\lambda) \in
\R^{n+1}, \quad e = [1]_{1 \leq i \leq m}\,.
\eeqn
Sei $x_0$ beliebig, z.B. $x_0 = 0$, und $\lambda_0 = \mbox{Max } \{
0\,,\:a_i^T x_0 - b_i\,,\;i = 1 , \ldots , m\}$, dann ist
$(x_0\,,\,\lambda_0)$ zul\"assig f\"ur $(^{\ast})$. Daher kann
$(^{\ast})$ mit dem obigen Algorithmus und anschlie\ss endem
Projektionsverfahren gel\"ost werden. Die optimale L\"osung
$(\widetilde{x}\,,\,\widetilde{\lambda})$ von $(^{\ast})$ existiert.
$\Omega = \{x \in I\!\!R^n \,|\, Ax \leq b\}$ ist leer, wenn
$\widetilde{\lambda} > 0$, andernfalls ist $\widetilde{x}$ zul\"assig.
%
%
%-- Rev. SIMPLEX-ALGORITHMUS ----------------------------------------
\newpage
\thispagestyle{empty}
\kopf
\vspace{-5mm}
\par
\ul{\bf Revidierter Simplexalgorithmus} f\"ur $\mbox{Min
  }\{c^T x, \; Ax = b, \; x \geq 0\}$, $ A\;(m,n)$--Matrix
vom Rang $m$ mit Spalten $a_j \,,\: m \leq n$,
$b = [\beta^1,\ldots , \beta^m]^T\,,\;c = [\gamma^1 , \ldots ,
\gamma^n]^T$.
\par
\ul{\bf Gegeben:} Ausgangsecke $x$ (ev. Basis geeignet
erg\"anzen,\\
$I_B(x) = (\rho_1, \ldots , \rho_m)$ Indexvektor bzw. Indexmenge der
Basisvariablen,\\
$I_N(x) = (\sigma_1, \ldots , \sigma_{n-m})$ Indexvektor bzw.  Indexmenge der
Nichtbasisvariablen\\
(Reihenfolge beliebig, aber dann fest).
\beqn
B := [a_j]_{j \in I_B(x)}, \; N := [a_j]_{j \in I_N(x)}, \;
c_B ;= [\gamma^j]_{j \in I_B(x)}, \; c_N := [\gamma^j]_{j \in
I_N(x)}.
\eeqn
%
{\bf Vorbereitung.} Berechne
\beqn \ba{.}{ll}
D = [d_j]_{1 \leq j \leq n-m} := B^{-1}N, & u := B^{-1}b, \\
y^T =[y_1 , \ldots , y_{n - m}] = c_N^T - c_B^TD, & z := -c^Tx.
\ea{.} \eeqn
\ul{\bf 1. Schritt:} (Welche Variable kommt in die Basis?)\\
Wenn $y \geq 0$, dann STOP ($x$ optimal), andernfalls berechne
\beqn \ba{.}{lll}
\nu := \mbox{ Min Arg Min } \{y_k \} & \mbox{(schnell)}, &
(a_{\sigma_{\nu}} \;\mbox{in die Basis }),\\
\nu := \mbox{Min Arg}\{ y_k , \; y_k < 0\}, &
\mbox{(sicher, Blandsche Regel, nur bei deg. Ecke)}.\\
\ea{.} \eeqn
\ul{\bf 2. Schritt:} (Welche Variable wird aus der Basis
entfernt?)\\
Wenn $\, d_{\nu} \leq 0$, dann STOP ( es existiert keine optimale
L\"osung), andernfalls berechne
\beqn
\mu := \mbox{ Min Arg Min } \{ u^k/d^k_{\nu}, \; d^k_{\nu} >
0\}, \quad (a_{\sigma_{\mu}}\;\mbox{aus der Basis })\,.
\eeqn
\ul{\bf 3. Schritt:} (Austauschschritt.)
\beqn
\ba{.}{ll}
\mbox{Setze} \; P = [p_{k \, l}] := \ba{[}{ll}
D & u\\
y^T &z\\
\ea{]}, &
\mbox{berechne } \, Q = [q_{k \, l}] \quad \mbox{gem\"a\ss}\\
q_{\mu \nu} = 1/p_{\mu \, \nu} \; \mbox{(Pivotelement)}, &
q_{k \, \nu} = - p_{k \nu }/p_{\mu \nu }, \; k \neq \mu
\; \mbox{(Pivotspalte)},\\
q_{\mu \, l} = p_{\mu \, l}/p_{\mu \nu}\:,\;l \neq \nu
\; \mbox{(Pivotzeile)}, &q_{k \, l} = p_{k \, l} - p_{k \, \nu}
p_{\mu \, l}/p_{\mu \nu} \; \mbox{(\"ubrige Elemente)}\,.
\ea{.} \eeqn
Setze
\beqn \ba{.}{ll}
D = [q_{k \, l}]_{1 \leq k \leq m\,,\,1 \leq l \leq n-m}, & u =  [q_{k,
  n-m+1}]_{1\leq k \leq m}\,,\\
y^T = [q_{m + 1,l}]_{1 \leq l \leq n-m}, &z =
q_{m+1\,,\,n-m+1}\,;\\
I_B(x) = (\rho_1 , \ldots , \rho_{\mu -
  1}\,,\,\sigma_{\nu}\,,\,\rho_{\mu+1}, \ldots , \rho_m)\,,\,
 &\\
I_N(x) = (\sigma_1 , \ldots , \sigma_{\nu -
  1}\,,\,\rho_{\mu}\,,\,\sigma_{\nu+1}, \ldots , \sigma_{n-m}) & \\
\ea{.} \eeqn
und kehre zum ersten Schritt des Verfahrens zur\"uck.
\beqn \ba{.}{ll}
\mbox
{
\begin{tabular}[l]{r|c}
& $I_N(x)$ \\
\hline
$I_B(x)$ & $P$
\end{tabular}
}
&
\ba{.}{l} \mbox{hei\ss t Simplextableau, $u$ enth\"alt die aktuellen
Werte der}\\
\mbox{Basisvariablen, $z$ den negativen aktuellen Wert der Zielfunktion.}
\ea{.}
\ea{.} \eeqn
$I_N(x)$ und $I_B(x)$ m\"ussen gespeichert werden, weil in $Ax = b$ Spalten
vertauscht werden.
%
%-- PRODUKTIONSPLANUNG ----------------------------------------------
\newpage
\thispagestyle{empty}
\kopf
\par
\ul{\bf Produktionsplanung mit Lagerhaltung.}
Es seien\\
$x(t)$ Produktion zur Zeit $t$ (st\"uckweise diff.-bar), $\dot{x} = u$;\\
$a(u)$ Kosten f\"ur die Produktion pro Zeiteinheit (ZE) bei der
Produktionsrate $u$\\
\hspace*{0,87cm}($a$ monoton wachsend und konvex);\\
$b(t)$ Nachfragerate zur Zeit $t$ $(b(t) \geq 0)$;\\
$z(t)$ Lagerumfang zur Zeit $t$;\\
$h \cdot z$ Lagerhaltungskosten/ZE bei einem Lager vom Umfang $z\: (h
\geq 0)$;\\
$c(u)$ Kosten/ZE bei Produktions\"anderung $u$.
\par
Bei der Produktionsplanung mit Lagerhaltung ergibt sich dann folgendes
Problem:
\beqn
\int\limits_0^T [ a(u(t)) + hz(t) + c(u(t))]\,dt = \mbox{Min }!
\eeqn
\beqn
\dot{z}(t) = u(t) - b(t)\,,\: z(0) = Z\,,\:z(t) \geq 0\,,\: u(t)
\geq 0\,.
\eeqn
(Vgl. Luenberger [69], S. 4, 234 ff., 237; Arrow u.a.: Studies in the
Math. Theory of Inventory and Production [1958].)
\par
Es sei nun $T < \infty$ und $c = 0$. Ferner sei
\beqn
s(t) = \int\limits_0^t b(\tau)\,d\tau - Z\,
\eeqn
dann gilt $z = x - s$, d.h.
\beqn
\mbox{Lagerumfang = Produktion - Nachfrage + anf�ngl. Lagerumfang,}
\eeqn
und man kann das Problem f\"ur die Produktion wie folgt formulieren:
\beqn
(^{\ast}) \quad J(u) = \int\limits_0^T [a(u(t)) + h\cdot(x(t) - s(t))]\,dt =
\mbox{Min }!, \eeqn
\beqn
x(t) = \int\limits_0^t u(\tau)\,d\tau \geq s(t)\,,\:u(t) \geq 0\,,\:
0 \leq t \leq T\,.
\eeqn
Es folgt nun $x(0) = 0$, $x$ monoton wachsend und o.B.d.A. $x(T) =
s(T)$ im Optimum, d.h. keine Ware \"ubrig am Ende des Planungszeitraums.
%
%
%
%-- MAXIMUMPRINZIP ----------------------------------------------
\newpage
\thispagestyle{empty}
\setcounter{equation}{0}
\kopf
\vspace{-8mm}
\par
\ul{\bf Euler--Lagrange--Gleichungen und Maximumprinzip von
Pontrjagin}
\par
Es seien $X = C^1([0,T];\R^m), \; U = C([0,T];\R^n), \; \Omega \subset \R^n$
fest gegeben, und es sei $(x^{\ast},u^{\ast})$ `regul\"are' L\"osung des
Maximumproblems
\beqn \ba{.}{l}
J(T,x,u) = \varphi(T,x(0),x(T)) + \int_0^TL(t,x(t),u(t))dt = \mbox{Max!}, \; x
\in X, \; u \in U,\\[2mm]
\dot{x}(t) = f(t,x(t),u(t)), \; u(t) \in \Omega, \; t \in (0,T),\\[2mm]
\psi (T,x(0),x(T)) = 0, \; \psi: \R^{2m+1} \supset D \to \R^p.
\ea{.} \eeqn
Dabei seien $L, \; f$ stetig differenzierbar in einer Umgebung des Graphen von
$(x^{\ast},u^{\ast})$, und $\varphi, \; \psi$ seien stetig differenzierbar in
einer
Umgebung von $(T,x^{\ast}(0),x^{\ast}(T))$. Ferner sei die Hamilton--Funktion
$H$ definiert durch
\beqn
H(t,x(t),u(t),y(t)) = L(t,x(t),u(t)) + y(t)^Tf(t,x(t),u(t)), \quad t \in (0,
T).
\eeqn
Dann existiert ein $y^{\ast} \in X$ und ein $z^{\ast} \in \R^p$ so, dass das
Quadrupel $(x^{\ast},u^{\ast},y^{\ast},z^{\ast})$ L\"osung des folgenden
Randwertproblems ist $(D_2f(x,y,z) := \nabla _yf(x,y,z)$, usw.):
\begin{equation}\label{11}
 \dot{x} = H_y(t,x,u,y), \quad \psi(T,x(0),x(T)) = 0,
\end{equation}
\begin{equation}\label{12}
\dot{y} = - H_x(t,x,u,y),
\end{equation}
\begin{equation}
y(0) = - D_2(\varphi  + z^T\psi)(T,x(0),x(T)), \;
y(T) = D_3(\varphi  + z^T\psi)(T,x(0),x(T)).
\end{equation}
Au\ss erdem gilt das Maximumprinzip von Pontrjagin (1959):
\begin{equation}\label{13}
u^{\ast}(t) = \mbox{Arg Max}_{u(t)\in\Omega}
\{H(t,x^{\ast}(t),u(t),y^{\ast}(t))\}, \quad t \in (0,T).
\end{equation}
Falls $T$ freie Ver\"anderliche ist, kommt hierzu noch
\begin{equation}\label{14}
D_1(\varphi + z^T\psi)(T,x(0),x(T)) + H(T,x(T),u(T),y(T)) = 0
\end{equation}
als notwendige Bedingung f\"ur das optimale $T^{\ast}$.
\par
(\ref{11}) folgt trivialerweise aus dem Extremalproblem, (\ref{12}) und
(\ref{14}) entsprechen den Euler--Gleichungen der Variationsrechnung, und
aus (\ref{13}) folgt bei hinreichender Glattheit von $H$ die notwendige
Bedingung
\beqn
D_3H(t,x^{\ast}(t),u^{\ast}(t),y^{\ast}(t)) = 0, \quad t \in (0,T),
\eeqn
falls $u^{\ast}(t)$ f\"ur alle $t \in (0,T)$ im Innern von $\Omega$ liegt.
\par
Falls $u^{\ast}$ nur st\"uckweise stetig ist, gilt (\ref{11}) und (\ref{12}) an
den Stetigkeitstellen von $u$.
%
%-- AUSGANGSECKE ----------------------------------------------------
\newpage
\thispagestyle{empty}
\kopf
\vspace{-5mm}
\par
\ul{\bf Berechnung einer Ausgangsecke} beim Projektionsverfahren f\"ur\\
$\mbox{Max }\{c^T x, \, Ax \leq b\}\,,\: A (m,n)$--Matrix,
Rang $(A) = n \leq m$.\\
\ul{\bf Gegeben:}
\beqn \ba{.}{l}
x \in \{x \in \R^n \,|\, Ax
\leq b\}\,,\; I =: B^{-1} = C = [c_i]_{1\leq i\leq n}\,,\: D:=
A = [d_i]_{1\leq i \leq n},\\[0,7ex]
 y^T = c^T\,,\: r = Ax -b \,,\: I_B(x) := 0 \in \R^n,
\: z = c^T x\,.\\
\ea{.} \eeqn
%
\ul{\bf 1. Schritt:} (Suchrichtung.)\\
Wenn $I_B(x) > 0$, dann STOP,andernfalls berechne
\beqn
\nu := \mbox{ Min Arg Max }\{ |y^k|, \: [I_B(x)]^k = 0\}.
\eeqn
Wenn $y^{\nu} > 0$, dann setze $\widetilde{d}_{\nu} = - d_{\nu}$.\\
Wenn $y^{\nu} < 0$, dann setze $\widetilde{d}_{\nu} = d_{\nu}$.\\
Wenn $y^{\nu} = 0$, dann setze
\beqn
\nu := Min \, Arg \, Min\{y_k, \; I_B(x)^k \neq o\},
\eeqn
%
\ul{\bf 2. Schritt:} (Schrittweite wie beim Projektionsverfahren.)\\


Wenn $\{i \in \{i, \ldots m\}\,|\, \widetilde{d}_{\nu}^i < 0 \} \neq
\emptyset$, dann berechne
\beqn
\tau = \mbox{Min }\{f(k) = r^k/\widetilde{d}_{\nu}^k
\;| \; \widetilde{d}^k_{\nu} < 0\}\,,\;\mu = \mbox{Min }\{k\,|\,f(k) =
\tau\}\,.
\eeqn
Setze
\beqn
x := x - \tau \,s \quad\mbox{und gehe zu Schritt 3.}
\eeqn
Wenn $\{i \in \{1 , \ldots , m\}\; | \; \widetilde{d}^i_{\nu} > 0\} \neq
\emptyset$, dann setze $\widetilde{d}^i_{\nu} := -
\widetilde{d}^i_{\nu}$ und gehe zu Schritt 2, andernfalls setze
$[I_B(x)]^{\nu} = 0$ und gehe zu Schritt 1.\\[1ex]
%
\underline{\bf 3. Schritt:} (Austauschschritt.)\\
Setze
\beqn
P :=
\left[
\begin{array}{ll}
D & r\\
h^T & z\\
\end{array}
\right]\,,
\eeqn
berechne $Q$ wie beim Projektionsverfahren und setze $[I_B(x)]^{\nu}
= \mu$ sowie
\beqn
D = [q_{k\, l}]_{1 \leq k\leq m\,,\,1 \leq l \leq n}\,,\; r =
[q_{k, n+1}]_{1 \leq k \leq m}\,,\; h^T = [q_{m+1,l}]_{1 \leq
   l \leq n}\,,\;z = q_{m+1\,,\,n+1}\,.
\eeqn
Gehe zu Schritt 1.\\[1ex]
Zur eventuellen Berechnung eines $x$ mit $Ax \leq b$ betrachte man
das Problem
\beqn
(^{\ast}) \quad \mbox{Min } \{\lambda \,|\, Ax - \lambda e \leq
b\,,\: \lambda \geq 0\}, \quad (x\,,\,\lambda) \in
\R^{n+1}, \quad e = [1]_{1 \leq i \leq m}\,.
\eeqn
Sei $x_0$ beliebig und $\lambda_0 = \mbox{Max } \{ 0\,,\:a_i^T
x_0 - b_i\,,\;i = 1 , \ldots , m\}$, dann ist
$(x_0\,,\,\lambda_0)$ zul\"assig f\"ur $(^{\ast})$. Daher kann
$(^{\ast})$ mit dem obigen Algorithmus und anschlie\ss endem
Projektionsverfahren gel\"ost werden. Die optimale L\"osung
$(\widetilde{x}\,,\,\widetilde{\lambda})$ von $(^{\ast})$ exisitert.
$\Omega = \{x \in I\!\!R^n \,|\, Ax \leq b\}$ ist leer, wenn
$\widetilde{\lambda} > 0$, andernfalls ist $\widetilde{x}$ Ecke von $\Omega$.
%
%
%
%-- MAXIMUMPRINZIP --------------------------------------------------
\newpage
\thispagestyle{empty}
\kopf
\vspace{-5mm}
\par
\ul{\bf Das Maximumprinzip von Pontrjagin (1959).}\\
 Es seien $ L: I\!\!R^{n+m+1} \to I\!\!R\,,\: f: I\!\!R^{n+m+1} \to
 I\!\!R^n\,,\: \phi: I\!\!R^n \to I\!\!R\,,\: \psi: I\!\!R^n \to
 I\!\!R^r$ stetig differenzierbar. Ferner sei $\Omega \in I\!\!R^m$
 gegeben und
$$
\begin{array}{lcl}
U = \{ u&:& [0\,,\,\top] \to \Omega\,,\: u \;\mbox{beschr\"ankt,
   st\"uckweise stetig,}\\
&\exists& x: [0\,,\,\top] \to I\!\!R^n\,,\; x(0) =
a\,,\:\stackrel{\cdot}{x} = f(x\,,\,u)\,,\: \psi (x(\top)) = 0 \}\,.
\end{array}
$$
 Es sei $(x^{\ast}\,,\,u^{\ast})$ `regul\"are' L\"osung des
 Extremalproblems
$$
(1) \qquad\mbox{Max } \{ \pi(x(\top)) + \int \limits_0^{\top}
L(x\,,\,u)\,dt | x(0) = a\,,\;\stackrel{\cdot}{x} =
f(x\,,\,u)\,,\;\psi(x(\top)) = 0\,,\: u \in U\}
$$
mit den Abk\"urzungen $f(x\,,\,u) = f(x(t)\,,\,u(t)\,,\,t)$ usw., und
es sei
$$
H(x\,,\,u\,,\,y) = L(x\,,\,y) + y^{\top}f(x\,,\,u)\,.
$$
Dann existiert ein $y^{\ast} : [0\,,\,\top] \to I\!\!R^n$ und ein
$z^{\ast} \in I\!\!R^r$, so dass gilt
$$
(2) \qquad \;\stackrel{\cdot}{x}\rule{0pt}{1ex}^{\ast} =
H_y(x^{\ast}\,,\,u^{\ast}\,,\,y^{\ast}) \,,\;x^{\ast}(0) = a\,,
$$
$$
\begin{array}{rcl}
(3) \qquad \;\stackrel{\cdot}{y}\rule{0pt}{1ex}^{\ast} =
-H_x (x^{\ast}\,,\,u^{\ast}\,,\,y^{\ast})&=&
-L_x(x^{\ast}\,,\,u^{\ast}) - (y^{\ast^{\top}} f)_x
(x^{\ast}\,,\,u^{\ast})\,,\\
y^{\ast}(\top) &=& \phi_x(x(\top)) + (z^{\ast^{\top}}
\psi)_x(x(\top))\,,
\end{array}
$$
%
$$
(4) \qquad \; H(x^{\ast}(t)\,,\,u^{\ast}(t)\,,\,y^{\ast}(t)\,,\,t) =
\mbox{Max}_{u(t) \in \Omega \,,\, u \in U}
H(x^{\ast}(t)\,,\,u(t)\,,\,y^{\ast}(t))\quad\mbox{f.\"u.
  }\:[0\,,\,\top]\,.
$$

(2) folgt trivialerweise aus (1), (3) sind die kanonischen
Euler--Gleichungen des Problems, und aus (4) folgt die notwendige
Bedingung
$$
H_u(x^{\ast}\,,\,u^{\ast}\,,\,y^{\ast}) = 0
\qquad\qquad\qquad\mbox{f.\"u.}\;[0\,,\,\top]\,,
$$
falls $\psi = 0$ und $u^{\ast}(t) \in \stackrel{\circ}{\Omega}
\;\forall \; t \in [0\,,\,\top]$.\\
Die Regularit\"atsbedingung von $(x^{\ast}\,,\,u^{\ast})$ entspricht
der Slater--Bedingung im Satz von Kuhn--Tucker.
%
%
%-- PRODUKTIONSPLANUNG ----------------------------------------------
\newpage
\thispagestyle{empty}
\kopf
\vspace{-8mm}
\par
\underline{\bf Produktionsplanung mit Lagerhaltung.} \\[0,7ex]
Es seien\\
$u(t)$ Produktionsrate zur Zeit $t$ (st\"uckweise stetig);\\
$a(u)$ Kosten f\"ur die Produktion pro Zeiteinheit (ZE) bei der
Produktionsrate $u$\\
\hspace*{0,87cm}($a$ monoton wachsend und konvex);\\
$b(t)$ Nachfragerate zur Zeit $t$ $(b(t) \geq 0)$;\\
$z(t)$ Lagerumfang zur Zeit (t);\\
$h \cdot z$ Lagerhaltungskosten/ZE bei einem Lager vom Umfang $z\: (h
\geq 0)\;\\
$c(u) Kosten/ZE bei Produktions\"anderung $u$.\\[1,3ex]
Bei der Produktionsplanung mit Lagerhaltung ergibt sich dann folgendes
Problem:
$$
\int\limits_0^{\top} [ a(u(t)) + hz(t) + c(u(t))]\,dt = \mbox{Min }!
$$
$$
\stackrel{\cdot}{z} = u - b\,,\: z(0) = Z\,,\:z(t) \geq 0\,,\: u(t)
\geq 0\,.
$$
(Vgl. Luenberger [69], S. 4, 234 ff., 237; Arrow u.a.: Studies in the
Math. Theory of Inventory and Production [1958].)\\

Es sei nun $\top = 1$ und $c = 0$. Ferner sei
$$
s(t) = \int\limits_0^t b(\tau)\,d\tau - Z\,.
$$
 Dann kann man das Problem wie folgt formulieren:
$$
J(u) = \int\limits_0^1 [a(u) + h(x - s)]\,dt = \mbox{Min }!
\vspace*{-0,7cm}
$$
$(^{\ast})$
\vspace*{-0,7cm}
$$
\hspace*{6,7cm} x(t) = \int\limits_0^t u(\tau)\,d\tau \geq
s(t)\,,\:u(t) \geq 0\,,\: 0 \leq t \leq 1\,.
$$
Es folgt $x(0) = 0$, $x$ monoton wachsend und $x^{\ast}(\top) =
s(\top)$ im Optimum.
%
\end{document}
