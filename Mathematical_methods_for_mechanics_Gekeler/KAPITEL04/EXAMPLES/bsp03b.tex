\documentclass[12pt,a4paper,twoside,leqno]{article}
\input aaformat
\newcommand{\Release}{10/05/05} %%%%%%%%%%%%%%
\begin{document}
{\large\bf Kontrolltheorie: Beispiel 3a \hfill E.Gekeler}
\par
\vspace{-0.5ex}
\hfill{\footnotesize\Release\ }
\par\hrule\par\vspace{2ex}
% ---------------------------------------------------------
{\bf Zermelosches Problem}
Vgl. {\sc Bryson-Ho}: Applied Optimal Control, S. 67.\\
Wie Beispiel 3, aber der Kozustand $y$ wird nicht eliminiert,
sondern das Gradientenverfahren angewendet.

({\sc Zermelo}sches Problem) In einem
$(x_1,x_2)$-Koordinatensystem soll ein Schiff mit konstanter Geschwindigkeit
$S$ relativ zum Wasser in m"oglichst kurzer Zeit $T$ von dem Punkt $A =
(a_1,a_2)$ zu dem Punkt $B = (b_1,b_2)$ fahren ($x = (x_1,x_2)$).
\[
\ba{.}{ll}
(x_1(t),x_2(t))  & \mbox{Position des Schiffes}\\
v_1(x) & \mbox{Str"omungsgeschwindigkeit in $x_1$-Richtung}\\
v_2(x) & \mbox{Str"omungsgeschwindigkeit in $x_2$-Richtung}\\
u & \mbox{Winkel der Schiffsachse zur $x_1$-Richtung (Kontrolle)}.
\ea{.}
\]

Zustandsgleichungen (Bewegungsgleichungen) $\dot{x} = [\nabla_y H]^T$:
%
\[
\dot{x}_1 = S \cos(u(t)) + v_1(x(t)),\quad
\dot{x}_2 = S \sin(u(t)) + v_2(x(t))
\]

{\bf (a)} {\sc Goh-Teo}-Transformation\\
Vgl.\ {\sc Craven}: Control and Optimization, \S 6.4.3.\\
Es wird die Substitution
\[
t = sT, \quad 0 < s < 1
\]
\par
\vspace{1mm}

gew"ahlt und die unbekannte Zeit $T$ als neue (abh"angige) Ver"anderliche
eingef"uhrt:
\[
\ba{.}{rclrcl}
X_3(s) &=& T, &
X_1(s) &=& x_1(sT) = x_1(sX_3(s)), \\
X_2(s) &=& x_2(sT) = x_2(sX_3(s)), &
U(s) &=& u(sX_3(s))\,.
\ea{.}
\]
\par
Neue Zustandsgleichungen
\[
\fbox{$
X'_1(s) = [S \cos(U(s)) + v_1(X(s))]X_3(s),\;
X'_2(s) = [S \sin(U(s)) + v_2(X(s))]X_3(s), \;
X'_3(s) = 0
$}\,.
\]

Zielfunktion und {\sc Hamilton}-Funktion
\[
\ba{.}{rcl}
J(X_3(1)) &=& \dis X_3(1) + (X_1(1) - B_1)^2 + (X_2 - B_2)^2 = \Min!\\[3mm]
H(X,Y,U) &=& \dis
Y_1X_3[S\cos(U) + v_1(X)] + Y_2X_3[S\sin(U) + v_2(X)]\\[3mm]
H(X,Y,U)_U &=& \dis
X_3S[Y_2\cos(U) - Y_1\sin(U)]\, .
\ea{.}
\]

Kozustandsgleichungen ({\sc Euler-Lagrange}-Gleichungen)
$Y' =  - [\nabla_X H]^T$:
\[
\fbox{$
\ba{.}{rcl}
Y'_1(s) &=& \dis
-X_3\left(Y_1\frac{\da v_1}{\da X_1} + Y_2\frac{\da v_2}{\da
X_1}\right)\\[4mm]
Y'_2(s) &=& \dis
- X_3\left(Y_1\frac{\da v_1}{\da X_2} + Y_2\frac{\da v_2}{\da
X_2}\right)\\[4mm] Y'_3(s) &=& \dis
Y_1[S\cos(U) + v_1(X)] + Y_2[S\sin(U) + v_2(X)]
\ea{.}
$}\; .
\]

Anfangsbedingungen f"ur $X$:
\[
X_1(0) = A_1, \quad X_2(0) = A_2, \quad X_3(0) \; \mbox{frei}\, .
\]
Randbedingungen f"ur $Y$:
\[
Y_3(0) = 0, \quad Y(1) = [\nabla_X p]^T(X(1))
= \ba{[}{c} 2(X_1(1) - B_1)\\ 2(X_2(1) - B_2)\\1\ea{]}\, .
\]
Es ergibt sich also ein Randwertproblem f"ur $X, \, Y$ bei festem $U$.

% ----------------------------------------------------------
Das nichtlineare Randwertproblem wird mit dem {\sc Matlab}-Programm
\verb/bvp4c.m/ gel"ost.  Als Startwert wird die direkte Fahrt von $A$ nach $B$
gew"ahlt.  Im folgenden einfachen Beispiel ist dies eine Strecke.

{\bf Beispiel}: Fahrt  von $A = (a_1,a_2)$ nach $B = (0,0)$ mit $v = (- S/2,0)$.
Winkel $\psi$ zwischen $\ol{AB}$ und $x_1$-Achse:
\[
\cos \psi = - \frac{a_1}{a^2_1 + a^2_2}, \;
\sin \psi = - \frac{a_2}{a^2_1 + a^2_2}.
\]
Kosinussatz:
\[
S^2 =  d^2 + \frac{S^2}{4} - dS \cos(\psi), \qquad
d = \dis \frac{S}{2}(\cos \psi + (3 + \cos^2  \psi)^{1/2})\,.
\]
Winkel $\phi $ zwischen Schiffsachse und Achse $\ol{AB}$:
\[
\frac{S^2}{4} = d^2 + S^2 - 2dS \cos \phi
\quad \Longrightarrow \quad
\cos \phi  = \dis \frac{d^2 + 3S^2/4}{2dS}.
\]

Die L"osung versagt wegen schlechter Kondition des Problems.
\par
\vspace{1mm}
% --------------------------------------------------
{\bf (b)} Gradientenverfahren mit `free terminal time'\\
{\bf (b1)} Zun"achst feste Zeit T.

Zustandsgleichungen (Bewegungsgleichungen) $\dot{x} = [\nabla_y H]^T$:
%
\[
\dot{x}_1 = S \cos(u(t)) + v_1(x(t)),\quad
\dot{x}_2 = S \sin(u(t)) + v_2(x(t))
\]

Zielfunktion und {\sc Hamilton}-Funktion
\[
\fbox{$
\ba{.}{rcl}
J(x(T)) &=& \dis (x_1(T) - B_1)^2 + (x_2(T) - B_2)^2 = \Min!\\[3mm]
H(x,y,u) &=& \dis
y_1[S\cos(u) + v_1(x)] + y_2[S\sin(u) + v_2(x)]\\[3mm]
H(x,y,u)_u &=& \dis
S[y_2\cos(u) - y_1\sin(u)]
\ea{.}
$}\; .
\]

Kozustandsgleichungen ({\sc Euler-Lagrange}-Gleichungen)
$\dot{y} =  - [\nabla_x H]^T$:
\[
\fbox{$
\ba{.}{rcl}
\dot{y}_1 &=& \dis
-\left(y_1\frac{\da v_1}{\da x_1} + y_2\frac{\da v_2}{\da x_1}\right)\\[4mm]
\dot{y}_2 &=& \dis
- \left(y_1\frac{\da v_1}{\da x_2} + y_2\frac{\da v_2}{\da x_2}\right)
\ea{.}
$}\; .
\]

Anfangsbedingungen f"ur $x$:
\[
x_1(0) = A_1, \quad x_2(0) = A_2\, .
\]
Randbedingungen f"ur $y$:
\[
y(T) = [\nabla_x p]^T(x(T))
= \ba{[}{c} 2(x_1(T) - B_1)\\ 2(x_2(T) - B_2)\ea{]}\, .
\]
\par
\vspace{1mm}
% -------------------------------------------------------
{\bf (b2)} Freie Zeit $T$.\\
Vgl. {\sc Dyer-McReynolds}, S. 125.\\
%
Zielfunktion

\[
\ba{.}{rcl}
J(x(T),T)
&=& \dis p(X(T),T) = T + (x_1(T) - B_1)^2 + (x_2(T) - B_2)^2 = \Min!\\[3mm]
\delta T &=& 1 + [2(x_1(T) - B_1)\,, \; 2(x_2(T) - B_2)]^T\cdot f(x(T),u(T),T)
\,.
\ea{.}
\]

Das Gradientenverfahren f"ur die Zeit $T$ wird "uber das Gradientenverfahren aus
(a) gest"ulpt.

\par
\vspace{1mm}
% --------------------------------------------------------
{\bf (b3)} Freie Zeit $T$ mit ``stop condition''\\
Zielfunktion mit geeignetem Gewicht $G$

\[
J(x(T),T) = \dis p(X(T),T) = T + G(x_2(T) - B_2)^2 = \Min!
\]

Stop-Bedingung
\[
r(x(T)) := x_1(T) - B_1 = 0\, .
\]
Dann ist
\[
\nabla_xr = \ba{[}{c}1\\0\ea{]}\,,\; \dot{r} =  f_1\,,\;
\ell = 0\,,\; p_T = 1\,, \;
\nabla_xp = \ba{[}{c}0\\ 2(x_2(T) - B_2)\ea{]}\,,
\]
also
\[
\fbox{$ \dis
\nabla_x\wi{p} =
\nabla_xp - \frac{1 + \nabla_xp\cdot f}{f_1}\ba{[}{c}1\\0\ea{]}
= \ba{[}{c}-(1 + \nabla_xp\cdot f)/f_1\\2(x_2(T) - B_2)\ea{]}
$}\; .
\]

Die Ergebnisse sind besser, wenn in $\nabla_xp$ das Gewicht $G$ gleich Eins
gesetzt wird.
%\newpage
%\bc
%\begin{minipage}{10.cm}
%   \epsfxsize=9.9cm
%   \epsffile{fig1.eps}
%\end{minipage}
%\ec
%\centerline{\epsfig{file=stab.eps,height=6cm},bbllx=-6cm,bblly=5mm}
\end{document}
