\documentclass[12pt,a4paper,twoside,leqno]{article}
\input aaformat
\newcommand{\Release}{10/05/05} %%%%%%%%%%%%%%
\begin{document}
{\large\bf Kontrolltheorie: Beispiel 9 \hfill E.Gekeler}
\par
\vspace{-0.5ex}
\hfill{\footnotesize\Release\ }
\par\hrule\par\vspace{2ex}
% ---------------------------------------------------------
{\bf Problem der Brachistochrone}\\
Die $x_2$-Achse weise nach unten. Setzt man
\begin{equation}\label{se1}
\dot{x}_1 = (2gx_2(t))^{1/2}\cos u(t)\,,\quad
\dot{x}_2 = (2gx_2(t))^{1/2}\sin u(t)\,,
\end{equation}
dann wird der Energieerhaltungssatz $mv(t)^2 = 2mgx_2(t)$ respektiert, und
das Problem lautet zusammen mit (\ref{se1})
\[
\ba{.}{rcl}
T &=& \Min!\,,\; x(0) = (0,0)\,,\; x(T) = a \,,\; 0 < a\,, \; 0 < b\,,\\[0.5ex]
%
0 &\leq& g(x_1,x_2)\,.
\ea{.}
\]
Die Kontrolle $u$ ist der Winkel
zwischen der Kurventangente und der $x_1$-Achse. Die Substitution $t = Ts$
ergibt das Problem
\[
\ba{.}{rcl}
T &=& \Min!\,,\; x(0) = (0,0)\,,\; x(1) = a\,,\\[0.5ex]
%
x'_1 &=& T(2gx_2(s))^{1/2}\cos u(s)\,,\;
x'_2 = T(2gx_2(s))^{1/2}\sin u(s)\,,\\[0.5ex]
0 &\leq& g(x_1,x_2)\,.
\ea{.}
\]
\end{document}
