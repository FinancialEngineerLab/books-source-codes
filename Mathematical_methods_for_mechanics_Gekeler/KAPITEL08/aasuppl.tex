\documentclass[12pt,a4paper,leqno]{book}
\input aaformat
\begin{document}
\addtolength{\abovedisplayshortskip}{-1ex}
\setlength{\fboxsep}{1ex}
\parskip1ex
%
\bc
{\bf Beispiele zum Babuska-Paradoxon}
\ec
%
{\sc Laplace}-Operator in kartesischen Koordinaten
\[
\Delta^2 w = w_{xxxx} + 2w_{xxyy} + w_{yyyy}
\]
{\sc Laplace}-Operator in Polarkoordinaten
\[
\ba{.}{rcl}
\Delta w &=& \dis w_{rr} + \frac{1}{r}w_r + \frac{1}{r^2}w_{\phi \phi } =: v\\
v_r &=& \dis w_{rrr} + \frac{1}{r}w_{rr} - \frac{1}{r^2}w_r
- \frac{2}{r^3}w_{\phi \phi } + \frac{1}{r^2}w_{r\phi \phi }\\[2ex]
v_{rr} &=& \dis
w_{rrrr} - \frac{2}{r^2}w_{rr} + \frac{1}{r}w_{rrr}
 + \frac{2}{r^3}w_r
 + \frac{6}{r^4}w_{\phi \phi } - \frac{4}{r^3}w_{r\phi \phi }
 + \frac{1}{r^2}w_{rr\phi \phi }\\[2ex]
v_{rr} &=& \dis w_{rrrr} - \frac{2}{r^2}w_{rr} + \frac{2}{r^3}w_r
 + \frac{1}{r}w_{rrr}
 + \frac{6}{r^4}w_{\phi \phi } - \frac{4}{r^3}w_{r\phi \phi } +
\frac{1}{r^2}w_{rr\phi \phi }\\[2ex]
v_{\phi \phi } &=& \dis w_{rr\phi \phi } + \frac{1}{r}w_{r\phi \phi } +
\frac{1}{r^2}w_{\phi \phi \phi \phi }\\[2ex]
\Delta^2w &=& \dis w_{rrrr} + \frac{2}{r}w_{rrr}- \frac{1}{r^2}w_{rr}
+ \frac{1}{r^3}w_r + \frac{2}{r^2}w_{rr\phi \phi }
 - \frac{2}{r^3}w_{r\phi \phi }
 + \frac{4}{r^4}w_{\phi \phi } + \frac{1}{r^4}w_{\phi \phi \phi \phi }
\ea{.}
\]
Es sei $\kappa = 1$ und $\Omega$ der Einheitskreis,
dann ist $\chi = 1$. Die jeweilige {\sc Poisson}-Zahl $\nu$ ergibt sich aus der
Gleichung
\[
v(1,\phi) = (\nu - 1)[\chi w_n(1,\phi) + p''(\phi)] \,.
\]

\par
{\bf Beispiel 1.}
\[
\ba{.}{rcl}
w(r,\phi ) &=& (1 - r^2)(5 - r^2)\,,\;\; w(1,\phi) = 0\\
w_r = w_n &=& -12r + 4r^3\\
- \Delta w(r,\phi ) = v(r,\phi ) &=& -(16r^2 - 24)
\ea{.}
\]

\[
\fbox{$
\Delta^2w          = 64, \;\;
v(1,\phi ) =  8, \;\;
w_n(1,\phi )        = - 8 ,\;\; \nu = 0
$} \; .
\]
\par
%%%%%%%%%%%%%%%%%%%%%%%%%%%%%%%%%%%%%%%%%%%%%%
{\bf Beispiel 2.}
\[
\ba{.}{rcl}
w(r,\phi ) &=& (1 - 2r^2)(5 - r^2)\,,\;\; w(1,\phi) = - 4\\
w_r = w_n &=& -22r + 8r^3\\
-\Delta w(r,\phi ) = v(r,\phi ) &=& -(32r^2 - 44)
\ea{.}
\]

\[
\fbox{$
\Delta^2w         =  128, \;\;
v(1,\phi ) = 12, \;\;
w_n(1,\phi )       =  - 14,\;\; \nu = 1/7
$} \; .
\]
%%%%%%%%%%%%%%%%%%%%%%%%%%%%%%%%%%%%%%%%%%%%%%%%%%%%%%%%%%%%
{\bf Beispiel 3.}
\[
\ba{.}{rcl}
w(r,\phi ) &=& (r^6 - 2r^4 - 4r^2)\cos(2\phi )/10\,,\;\;
w(1,\phi) = - \cos(2\phi)/2\\[0.5ex]
w_r = w_n &=& (6r^5 - 8r^3 - 8r)\cos(2\phi )/10\,,\;\;
w_n(1,\phi) = - 10\cos(2\phi)/10;\\[0.5ex]
-\Delta w(r,\phi ) = v(r,\phi ) &=& -(32r^4 - 24r^2)\cos(2\phi )/10\\[0.5ex]
p(\phi) &=&  -\cos(2\phi)/2\,,\;\; p''(\phi) = 2\cos(2\phi)\\[0.5ex]
w_n(1,\phi ) + p''(\phi ) &=& 10\cos(2\phi )/10
\ea{.}
\]
\par
\[
\fbox{$
\ba{.}{ll}
\Delta^2w         =  384r^2\cos(2\phi)/10, \;\;
v(1,\phi ) = -8\cos(2\phi)/10, \\[0.5ex]
w_n(1,\phi ) + p''(\phi )      =  10\cos(2\phi )/10\,,\;\; \nu = 1/5
\ea{.}
$} \; .
\]
%%%%%%%%%%%%%%%%%%%%%%%%%%%%%%%%%%%%%%%%%%%%
\newpage
{\bf Beispiel 4.}
\[
\ba{.}{rcl}
w(r,\phi ) &=& (2r^6 - 5r^4)\cos(2\phi )/12, \;\;
w(1,\phi) = - 3\cos(2\phi)/12\\[0.5ex]
w_r = w_n &=& (12r^5 - 20r^3)\cos(2\phi )/12,\;\;
w_n(1,\phi) = -8\cos(2\phi)/12\\[0.5ex]
- \Delta w(r,\phi ) = v(r,\phi ) &=& - (64r^4 - 60r^2)\cos(2\phi )/12\\[0.5ex]
p(\phi) &=& -3\cos(2\phi)/12,\;\; p''(\phi) = 12\cos(2\phi)/12\\[0.5ex]
w_n(1,\phi ) + p''(\phi ) &=& 4\cos(2\phi )/12
\ea{.}
\]
\par
\[
\fbox{$
\ba{.}{ll}
\Delta^2w =  64r^2\cos(2\phi), \;\;
v(1,\phi ) = - 4\cos(2\phi)/12, \\[0.5ex]
w_n(1,\phi ) + p''(\phi ) =  4\cos(2\phi )/12, \;\;
\nu  = 0
\ea{.}
$} \; .
\]

\renewcommand{\arraystretch}{1.5}
Maximaler Fehler:
\bc
\begin{tabular}{|c|c|c|c|} \hline
 & $\chi = 0$ & $\chi = 1$, 5 Schritte& $\chi = 1$, 10 Schritte  \\ \hline
Beisp. 1 & 2.0534   & 0.1692  & 0.0812 \\ \hline
Beisp. 2 & 3.1077   & 0.1865  & 0.1133 \\ \hline
Beisp. 3 & 0.0190   & 0.0061  & 0.0061 \\ \hline
Beisp. 4 & 0.0188   & 0.0076  & 0.0076 \\ \hline

\end{tabular}
\ec
\renewcommand{\arraystretch}{1}


\end{document}
