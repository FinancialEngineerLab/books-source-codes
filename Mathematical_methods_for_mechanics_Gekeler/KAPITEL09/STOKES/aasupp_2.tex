\documentclass[12pt,a4paper,USenglish,twoside]{book}
\input aaformat_e
%%%%%%%%%%%% Springer-Format %%%%%%%%%%
%\documentclass[envcountchap,USenglish,natbib]{svmono}
%%%%%%%%%%%%%%%%%%%%%%%%%%%%%
\usepackage{float}
\usepackage{multicol}
\usepackage{graphics}
%\input aaformat_s
\smartqed
\begin{document}
\setlength{\fboxsep}{1ex}
\newcommand{\Release}{22/10/09}
\addtolength{\abovedisplayshortskip}{-1ex}
\setlength{\fboxsep}{1.5ex}
\parskip0.5ex
\parindent0ex
\mainmatter
%
{\large\bf Supplements 2 to Chapter IX\hfill E.\ Gekeler
}
\par
\vspace{-1mm}
\hfill{\footnotesize\Release\ }
\par
\vspace{-2mm}
\rule{\textwidth}{1pt}
\par\vspace{2ex}
%%%%%%%%%%%%%%%%%%%%%%%%%%%%%%%%%%%%%%%%%%%%%%%%%%%%%%%%%%%
{\bf Case Study: Navier-Stokes equations}
\par
%
Transient NS equations (Gresho, p. 450)
%
\[
\rho \frac{D\u}{D t} \equiv 
\rho \left(\frac{\da \u}{\da t} +  (\grad \u)\u\right) 
= \mu \Delta \u - \nabla p 
\] 
Non-dimensional form:\\
$u_0$ char. velocity, $L$ char. length, $\tau$ char. time, $P_c$ char. pressure\\
{\sc Reynolds} number Re = $\rho u_0 L/\mu$ 
(ratio of advective to diffusive momentum transport). Then
\[
\frac{\da \u}{\da t} +  Re \,(\grad \u)\u 
=   \Delta \u - \nabla p \quad \quad
\text{for $\tau = \rho L^2/\mu$ (with $P_v =  \mu u_0/L$)}
\] 
%
\[
\frac{\da \u}{\da t} +  (\grad \u)\u 
= \nu\, \Delta \u - \nabla p \quad \quad
\text{for $\tau = L/u_0$ (with $P_c = \rho u^2_0$)}\,.
\] 
Weak form with $\nu = 1/Re$\,:
\begin{equation} \label{e0811.4}
\fbox{$\
\ba{.}{llllll}
(\v,\da \u/\da t)\\
 + a(\v,\u)  + c(\v,\u,\w)&-&(\div \v,p_{\rm spec}) &=& (\v,\f)
+ (\v,\ul{\sigma}_n(\u,p_{\rm spec}))_{\Gamma}\,, & \v \in \V \\[0.5ex]
&-&(\div \v,q) &=& 0\,, & q \in \Q\\[0.5ex]
&&(\v,\ul{\sigma}_n(\u,p_{\rm spec}))_{\Gamma}
&=& -(p_{\rm spec}\v\cdot \n)_{\Gamma} + \nu(\v\cdot (\grad \u)\n)_{\Gamma} 
\ea{.}
$}
\end{equation}
Pressure phase (Quartapelle 1982)
\[
\ba{.}{llll} \dis
\left(\v \,,\, \frac{\u^{n+1} - \u^{n+1/2}}{\Delta t}\right)
- (\div \v, p^{n+1}) &=& \dis \int\v \cdot \c\, d\Gamma_2\,, & \forall \v \,,\;
\n\cdot\v|_{\Gamma_1} = 0\\[2ex]
(\div \u^{n+1},q) &=& 0\,, & \n\cdot\u^{n+1}|_{\Gamma_1}  = \n\cdot \b\,.\\[2ex]
(-p\,I + \nu \,\grad \u)\n &=& \c

\ea{.}
\]
For the linear {\sc Stokes} equation $-\Delta \u + \grad p = \f$ follows the
pressure equation 
\[
-\Delta p = - \div f\,,\quad 
\frac{\da p}{\da n} = \Delta \u \cdot \n + \f\cdot \n \;\; \text{on} \;\; \Gamma
\]
which is equivalent with
\[
\w = -\grad p \,,\quad \div\w = - \div \f
\]
\par\vspace{1ex}\hrule\vspace{1ex}
\newpage
Manuskript p. 204:
\begin{example}\label{b2} (Application of the complementary energy principle.)
Consider the simple {\sc Dirichlet} problem: Find $u^* \in H^1_0(\Omega)$ such that
%
\begin{equation}\label{e2}
u^* = \arg\min\left\{\frac12 \int_{\Omega}\big[|\grad u|^2 \, dx - f\,u\big]\, dx\,;
\; u \in H^1_0(\Omega)\right\}
\end{equation}
\index{Dirichlet problem}
%
where $u = 0$ on the entire boundary of the domain $\Omega$\,; see Sects.\ 
\ref{sec0107}, \ref{sec0901} for notations. We insert the relation
%
\[
\frac12\int_{\Omega}|\grad u|^2\, dx = \sup_{\v\in (\L^2(\Omega))^2}
\int_{\Omega}\left[\v \cdot \grad u - \frac12 |\v|^2\right]\, dx
\]
and recall that
%
\begin{equation}\label{e3}
\dis \int_{\Omega}\grad u \cdot \v \,dx  =  - \int_{\Omega}u \,\div \v\, dx
\end{equation}
%
in the present case of homogeneous {\sc Dirichlet} boundary conditions. Then we get 
again the saddlepoint problem corresponding to (\ref{e2}) in two alternative forms: Find $(u^*,\v^*) \in \U\times \V$ such that
%
\begin{equation}\label{e4}
\ba{.}{lll}
(u^*,\v^*) &=& \dis \arg\min{}_u\sup{}_{\v}
\left\{\int_{\Omega}\left[-\frac12|\v|^2 - f\,u  + \grad u \cdot \v\right] dx\,,
\; u \in \U\,,\; \v \in \V \right\}\\
%
 &=& \dis \arg\max{}_{\v} \inf{}_u
\left\{\int_{\Omega}\left[-\frac12|\v|^2 - f\,u  - u \div \v\right] dx\,,
\; u \in \U\,,\; \v \in \V \right\}\,.
\ea{.}
\end{equation}
%
where $\U = \H^1_0(\Omega)$ and $\V = \L^2(\Omega) \times \L^2(\Omega)$\,.
\index{saddlepoint problem}
Both representations are equivalent in the case where a saddlepoint exists. Again  
the infimum over $u$ must exist finitely and the {\em dual maximum problem} is readily 
obtained from the second form (written as minimum problem): Find $\v^* \in \W$ such that
%
\begin{equation}\label{e5}
\v^* = \arg\inf_{\v \in \W} \int_{\Omega}|\v|^2\, dx\,, \;\; 
\W   = \left\{\v \in \V\,,\; \div \v + f = 0\right\}\,.
\end{equation}
\index{principle!complementary energy}
The saddlepoint $(u^*,\v^*)$ is also characterized by the two variational equations
associated, e.g., to the first form in (\ref{e4}) where we use (\ref{e3}) once more
\[
\ba{.}{llll}
\dis \int_{\Omega}\big[ \v\cdot \w -  \grad u \cdot \w \big] dx &=& 0\,,&
\forall \; \w \in \V\\[1ex]
\dis \int_{\Omega}\big[ u \div  \v \,dx + f\,u\big]dx  &=& 0\,, &
\forall \; u \in \U\,.
\ea{.}
\]
\index{Dirichlet problem!dual} 
Passing to the associated boundary value problem of this {\em dual} form of the {\sc 
Dirichlet} problem we get two {\em separated} first order equations
%
\begin{equation}\label{e6}
\fbox{$\
\v = \grad u\,, \;\; \div \v + f = 0\,,\;\; u \in H^1_0(\Omega)
$}
\end{equation}
which of course are also found directly by decomposing $- \Delta u = - \div \grad u = 
f$\,. The dual form is well-suited in cases where the {\em gradient} of the unknown 
function $u$ is more important than $u$ as maybe in stationary heat distribution of 
a disc etc., see for instance Example \ref{b0902.2}.)
\end{example}
\par
Manuskript, p. 433:\\
{\bf (f) Pressure Poisson equation}
In solving {\sc Navier-Stokes} equations, the results for pressure $p$ are 
frequently not very convincing. Rather it is calculated by a separate
{\sc Poisson} problem which however has only {\sc Neumann} boundary conditions
such that a reference value of $p$ must be specified somewhere in the domain
$\Omega$ or $\dis \int_{\Omega}p\, dV = 0$ must be required for
normalization but nevertheless the problem is and remains {\em unstable}.
\par
{\bf (f1)} For the computation of  $p$ by the flow field $\u$\,,
remember that
\[
\Delta \v = \div (\grad \v)^T = [\grad \div \v]^T\,.
\]
and apply divergence to the {\em linear} {\sc Stokes} equation,
$- \Delta \u + \grad p = \f\,$, yielding
\enlargethispage{6ex}
\[
- \div \Delta \u + \div \grad p = \div \f\,.
\]
But
\[
\div \Delta \u = \div \div \grad \u = \div \grad \div \u
= \Delta \div \u = 0\,,
\]
because $\div \u = 0$ by assumption hence
\[
\fbox{$\
\Delta p = \div \f
$} \; .
\]
On the other side, multiplying the {\sc Stokes} equation on the boundary $\Gamma$ by the 
normal vector $\n$, yields the condition
$
\Delta \u \cdot \n + \da p/\da \n = \f \cdot \n
$\,.
On summarizing we obtain a {\sc Neumann} problem for the pressure $p$\,,
%
\[
\fbox{$\
\ba{.}{rcll}
\Delta p &=& \div \f &\mbox{in} \; \Omega\\
\dis \frac{\da p}{\da n} &=&
 \Delta \u\cdot \n + \f\cdot \n  & \mbox{on} \; \da \Omega
\ea{.}
$}\;.
\]
%
\par
%%%%%%%%%%%%%%%%%%%%%%%%%%%%%%%%%%%%%%%%%%%%%%%%%%%%%%%%%%%%%%%%%%%%
{\bf (f2}) \cite{Sohn}. Reconsider the homogeneous non-dimensional {\sc Navier-Sto\-kes} 
equation (\ref{e0811.5e}),
%
\begin{equation}\label{e0811.16}
\frac{\da\u}{\da t} + (\nabla \u)\u
 - \frac{1}{R_e}\Delta \u + \nabla p = \ul{0}\,,\;\; \div \u = 0\,. 
\end{equation}
%
An alternative pressure {\sc Poisson} equation is obtained by differentiating
the first momentum equation in (\ref{e0811.16}) with respect to $x$ and the second 
with respect to $y$\,, and adding them. The result can be written as
%
\begin{equation}\label{e0811.17}
\Delta p = - \nabla[(\nabla \u)\u] + R_e^{-1}[(\Delta u)_x + (\Delta v)_y]\,.
\end{equation}
\index{pressure Poisson equation}
%
Let $q$ be an arbitrary test function for $p$ and apply {\sc Green}'s theorem then we 
obtain the weak form where $\n$ is the unit vector normal to boundary $\Gamma = \da 
\Omega$ pointing outward from $\Omega$\,,
%
\begin{equation}\label{e0811.18}
\ba{.}{lll} \dis
\int_{\Omega}\nabla q\cdot \nabla p \, d\Omega
&=& \dis - \int_{\Omega}\nabla q \cdot (\nabla \u)\u\, d\Omega
+ R_e^{-1}\int_{\Omega}\nabla q\cdot \Delta \u\, d \Omega\\[2ex]
&+& \dis  \int_{\Gamma}q\n\cdot[(\nabla \u)\u + \nabla p
 - R_e^{-1}\Delta \u]\, d \Gamma\,.
\ea{.}
\end{equation}
%
Substituting once more the momentum equation (\ref{e0811.16}) into the line integral
yields
%
\begin{equation}\label{e0811.19}
\int_{\Omega}\nabla q\cdot \nabla p \, d\Omega
= - \int_{\Omega}\nabla q \cdot (\nabla \u)\u\, d\Omega
+ R_e^{-1}\int_{\Omega}\nabla q\cdot \Delta \u\, d \Omega
-\int_{\Gamma}q \frac{\da \u}{\da t} \cdot \n\, d\Gamma\,.
\end{equation}
%
The line integral involving the time rate of change of the velocities will 
vanish at all solid boundaries or when steady-state solutions are sought. It is non-zero 
only along open boundaries in time-dependent flows, or if the flow is excited by the 
time-varying motion of a wall \cite{Sohn}. On the other side, the evaluation of the 
right side of (\ref{e0811.19}) requires the second derivatives of the velocity 
components (which, by the way, are known globally only as continuous functions). To 
overcome this difficulty at least locally, \cite{Sohn} 
proposes a least squares approximation of the first derivatives on the ansatz functions 
in case they are linear or bilinear and gets appealing results for the latter case.   
\end{document}
