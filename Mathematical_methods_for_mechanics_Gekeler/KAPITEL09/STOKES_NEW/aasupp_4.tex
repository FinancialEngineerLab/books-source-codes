\documentclass[12pt,a4paper,USenglish,twoside]{book}
\input aaformat_e
%%%%%%%%%%%% Springer-Format %%%%%%%%%%
%\documentclass[envcountchap,USenglish,natbib]{svmono}
%%%%%%%%%%%%%%%%%%%%%%%%%%%%%
\usepackage{float}
\usepackage{multicol}
\usepackage{graphics}
%\input aaformat_s
\smartqed
\begin{document}
\setlength{\fboxsep}{1ex}
\newcommand{\Release}{11/10/06}
\addtolength{\abovedisplayshortskip}{-1ex}
\setlength{\fboxsep}{1.5ex}
\parskip0.5ex
\parindent0ex
\mainmatter
%
{\large\bf Supplements 4 to Chapter IX\hfill E.\ Gekeler
}
\par
\vspace{-1mm}
\hfill{\footnotesize\Release\ }
\par
\vspace{-2mm}
\rule{\textwidth}{1pt}
\par\vspace{2ex}
%%%%%%%%%%%%%%%%%%%%%%%%%%%%%%%%%%%%%%%%%%%%%%%%%%%%%%%%%%%
{\bf Case Study: Convection term for post processor}
\par
%
{\bf (b) Trilinear Form} with shape functions.\\
$
P(q,\v,\w) = \grad q \cdot [\grad \v]\w
$
\[
\u \simeq \ \ba{[}{r}\Phi(x,y)^TU_1\\ \Phi(x,y)^TU_2\ea{]}\,,\;
\]
\par\vspace{1ex}
\[
N(q,\v,\w) \simeq \
Q^T [(\Phi_x^TV_1)\Phi_x\,\Phi^TW_1 + (\Phi_y^TV_1)\Phi_x\,\Phi^TW_2
     +(\Phi_x^TV_2)\Phi_y\,\Phi^TW_1 + (\Phi_y^TV_2)\Phi_y\,\Phi^T W_2]
\]
\par
\[
\Phi_x = \Psi_{\xi}\xi_x + \Psi_{\eta}\eta_x\,,\;\;
\Phi_y = \Psi_{\xi}\xi_y + \Psi_{\eta}\eta_y\,,
\]
\par\vspace{1ex}
\[
\int_TN(q,\v,\w) \simeq
Q^T[A(V_1)W_1 + B(V_1)W_2 + C(V_2)W_1 +  D(V_2)W_2]
\]
where
\begin{equation}\label{e2}
\ba{.}{lll}
A(V_1) &=& \dis J\int_S\big((\Psi_{\xi}\xi_x + \Psi_{\eta}\eta_x)^TV_1\big)
(\Psi_{\xi}\xi_x + \Psi_{\eta}\eta_x)\,\Psi^T\, d\xi d\eta\,, \\[1ex]
B(V_1) &=& \dis J\int_S\big((\Psi_{\xi}\xi_y + \Psi_{\eta}\eta_y)^TV_1\big)
(\Psi_{\xi}\xi_x + \Psi_{\eta}\eta_x)\,\Psi^T\, d\xi d\eta\\[1ex]
C(V_2) &=& \dis J\int_S\big((\Psi_{\xi}\xi_x + \Psi_{\eta}\eta_x)^TV_2\big)
(\Psi_{\xi}\xi_y + \Psi_{\eta}\eta_y)\, \Psi^T\, d\xi d\eta \,,\\[1ex]
D(V_2) &=& \dis J\int_S\big((\Psi_{\xi}\xi_y + \Psi_{\eta}\eta_y)^TV_2\big)
(\Psi_{\xi}\xi_y + \Psi_{\eta}\eta_y)\,\Psi^T\, d\xi d\eta
\ea{.}
\end{equation}
Now we have the representation
\[
\Psi_{\xi} = \ \ba{[}{rrr}
 -3 & 4 & 4\\
 -1 & 4 & 0\\
 0 & 0 & 0\\
 4 & -8 & -4\\
 0 & 0 & 4\\
0 & 0 & -4\ea{]}
\ba{[}{r}1\\ \xi \\ \eta\ea{]}
 =: [A_1\,,\, A_2\,,\, A_3]\ \ba{[}{r}1\\ \xi \\ \eta\ea{]}
\]
\[
\Psi_{\eta} = \ \ba{[}{rrr}
 -3 & 4 & 4\\
  0 & 0 & 0\\
 -1 & 0 & 4\\
 0 & -4 & 0\\
 0 & 4 & 0\\
 4 & -4 & -8\ea{]}\
\ba{[}{r}1\\ \xi \\ \eta\ea{]}
 =: [B_1\,,\, B_2\,,\, B_3]\ \ba{[}{r}1\\ \xi \\ \eta\ea{]}
\]
Accordingly, only the both matrices $P\,,Q$ are to be calculated:
\[
M := S_4 = \int_S \Psi\,\Psi^T\, d\xi d\eta\,, \;\;
P = \int_S\xi \Psi\,\Psi^T\, d\xi d\eta\,, \;\;
Q = \int_S\eta \Psi\,\Psi^T\, d\xi d\eta\,,
\]
Then we obtain
\[
\ba{.}{lll}
\dis \int_S(\Psi_{\xi}V)\Psi\Psi^T \,d\xi d\eta
&=& (A_1^TV)M + (A_2^TV)P + (A_3^TV)Q\\[1ex]
\dis \int_S(\Psi_{\eta}V)\Psi\Psi^T \,d\xi d\eta
&=& (B_1^TV)M + (B_2^TV)P + (B_3^TV)Q\\[1ex]
\ea{.}
\]
and the integrals (\ref{e2}) are linear combinations of these both integrals with the 
corresponding arguments $V_1$ resp.\ $V_2$
\[
\ba{.}{lll}
A(V_1) &=& [(A^T_1y_{31} + B^T_1y_{12})V_1]M + [(A^T_2y_{31} + B^T_2y_{12})V_1]P
+ [A^T_3y_{31} + B^T_3y_{12})V_1]Q\\[0.5ex]
%
B(V_1) &=& [(A^T_1x_{13} + B^T_1x_{21})V_1]M + [(A^T_2x_{13} + B^T_2x_{21})V_1]P
+ [(A^T_3x_{13} + B^T_3x_{21})V_1]Q\\[0.5ex]
%
C(V_2) &=& [(A^T_1y_{31} + B^T_1y_{12})V_2]M + [(A^T_2y_{31} + B^T_2y_{12})V_2]P
+ [A^T_3y_{31} + B^T_3y_{12})V_2]Q\\[0.5ex]
%
D(V_2) &=& [(A^T_1x_{13} + B^T_1x_{21})V_2]M + [(A^T_2x_{13} + B^T_2x_{21})V_2]P
+ [(A^T_3x_{13} + B^T_3x_{21})V_2]Q
\ea{.}
\]
{\bf(c)}
The nonlinear problem may be solved by a simple iteration method or by {\sc Newton}'s 
method. The latter allows higher {\sc Reynolds} numbers $Re = 1/\nu$. Also, in {\sc 
Newton}'s method the gradient of the nonlinear part must be calculated. From {\bf (b)}
\[
\ba{.}{l}
\grad_{U_1,U_2}
\, \ba{[}{rr} A(U_1) & B(U_1)\\ C(U_2) & D(U_2)\ea{]}
\ba{[}{r}U_1\\ U_2\ea{]}
= \ \ba{[}{rr} A(U_1) & B(U_1)\\ C(U_2) & D(U_2)\ea{]}
\\[3ex]
 + \ \ba{[}{cc} \grad_{U_1}A(U_1)U_1 + \grad_{U_1}B(U_1)U_2 & \text{Null} \\ 
     \text{Null} & \grad_{U_2}C(U_2)U_1 + \grad_{U_2}D(U_2)U_2\ea{]}
\ea{.}     
\]
\par
{\bf (d) Postprozessor}
Let $\u = (u,v)$ be the velocity field then the streamlines $z$ satisfy
%
\[
\int _{\Omega}\nabla \delta z\, \nabla z
= \int_{\Omega}\delta z\, w  + \int_{\Gamma}\delta z \, z_{\n}
\]
\[
\int_T\phi_iw = \int_T\phi_i(v_x - u_y) + \int_{\Gamma}\delta z \, z_{\n}
\]
\[
\ba{.}{l}
\dis \int_T\delta z\, w \,dxdy\\[1ex]
\simeq \dis
JZ\int_S\Psi(\Psi_{\xi}\xi_x + \Psi_{\eta}\eta_x)^T\,d\xi d\eta V
- JZ \int_S\Psi(\Psi_{\xi}\xi_y + \Psi_{\eta}\eta_y)^T\,d\xi d\eta U\\[1ex]
= \dis
JZ \left[\int_S\Psi\Psi_{\xi}^T\,d\xi d\eta\right](V\xi_x - U\xi_y)
+
JZ\left[\int_S\Psi\Psi_{\eta}^T\,d\xi d\eta\right](V\eta_x - U\eta_y)
\ea{.}
\]
Accorcingly, the matrices
\[
\int_S\Psi\Psi_{\xi}^T\,d\xi d\eta\,,\;\;
\int_S\Psi\Psi_{\eta}^T\,d\xi d\eta
\]
are to be calculated.
%
\[
z_x = - v\,,\;\; z_y = u\,,\;\; \grad z\cdot \n = -v\,n_1 + u\,n_2.
\]
%
\[
\xi _x = y_{31}/J, \;\; \xi _y = x_{13}/J, \;\;
\eta _x = y_{12}/J, \;\; \eta _y = x_{21}/J\,.
\]

\end{document}
