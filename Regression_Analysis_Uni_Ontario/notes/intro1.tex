\documentclass[landscape]{slides}
%\usepackage{defscup}

\usepackage[usenames,dvipsnames]{color}
\usepackage{graphicx}
\usepackage{times}
\usepackage{amssymb}
\usepackage{fancyvrb}
\usepackage{float}
\usepackage[stable]{footmisc}
\setlength\hoffset{0in} \setlength{\textwidth}{9in}
\setlength\voffset{-1in} \setlength{\textheight}{7.2in}
%\def\blue#1{{\color{blue} #1}}
\def\eps{\varepsilon}
\newcommand{\tab}{\hspace{5mm}}
\newcommand{\fc}{\footnote} 
\DefineVerbatimEnvironment%
  {verbcode}{Verbatim}{xleftmargin=2.5mm}
\DefineVerbatimEnvironment%
  {verbout}{Verbatim}{xleftmargin=2.5mm}
\DefineVerbatimEnvironment%
  {verbrun}{Verbatim}{xleftmargin=2.5mm} 
\VerbatimFootnotes

\newcommand{\red}[0]{\color{Red}}
\newcommand{\blue}[0]{\color{Blue}}
\newcommand{\header}[1]{
    \red
    #1\\
    \rule[1in]{\textwidth}{.01in}
    \vspace{-1.5in}
    \blue
}


\begin{document}

%\addcontentsline{toc}{chapter}{Introduction}

\begin{slide}
\header{Introduction}
\begin{itemize} 
\item The goal of data analysis is to highlight the information 
in given data. 
\item Exploratory Data Analysis -- a graphical and
numerical exploration of the  data 
\item Modelling -- using
mathematics, in conjunction with subject-area theory, 
to explain the data
\item Computing 
\item Reporting -- presenting results clearly, using 
graphs as appropriate
\end{itemize} 
\end{slide}

\begin{slide}
\header{Data Frames in R} 
\begin{itemize}
\item
An important class of R object is the data frame. R uses data 
frames to
store rectangular arrays in which the columns may be vectors of
numbers or factors or text strings.  A
data frame is like a matrix, where the rows are observations
and the columns are variables.
\item 
For example, try
\begin{verbatim}
> head(cars) # shows the first rows of the data frame cars
  speed dist
1     4    2
2     4   10
3     7    4
4     7   22
5     8   16
6     9   10
\end{verbatim}
\end{itemize}
\end{slide}

\begin{slide}
\header{Use of \texttt{summary}}
\begin{itemize}
\item 
\begin{verbatim}
> summary(cars)   # computes summary statistics
     speed           dist
 Min.   : 4.0   Min.   :  2.00
 1st Qu.:12.0   1st Qu.: 26.00
 Median :15.0   Median : 36.00
 Mean   :15.4   Mean   : 42.98
 3rd Qu.:19.0   3rd Qu.: 56.00
 Max.   :25.0   Max.   :120.00
\end{verbatim}
\item We can immediately see that the range of speeds (first 
column)
is from 4 mph to 25 mph, and that the range of distances (second
column) is from 2 feet to 120 feet.
\end{itemize}
\end{slide}

\begin{slide}
\header{Entry of data at the command line}\label{dataentry}
\begin{itemize}
\item
The following table gives, 
the distance that
the band traveled when released,
for each amount by which an
elastic band is stretched over the end of a ruler, 
\begin{verbatim}
Stretch (mm)  Distance (cm)
      46      148
      54      182
      48      173
      50      166
      44      109
      42      141
      52      166
\end{verbatim}
\item We can use \verb+data.frame()+ to input these (or other) 
data directly at the command line. We will assign the name
\verb+elasticband+ to the data frame:
\end{itemize}
\end{slide}
\begin{slide}
\header{Using data.frame() at the command line}
\begin{itemize}
\item[ ]
\begin{verbatim}
elasticband <- data.frame(stretch=c(46,54,48,50,44,42,52),
               distance=c(148,182,173,166,109,141,166))
\end{verbatim}
\end{itemize}
\end{slide}
\end{document}




\subsection{Retaining objects between sessions}

Note the commands
\verb!save.image()! for saving the current workspace,
\verb!rm(list=ls())!  for clearing the workspace, \verb!setwd()! for
changing the working directory, and \verb!load()! for loading an
existing workspace.  

Type
\verb+ls()+ to get a complete list of objects. Then remove unwanted
objects using
\begin{verbatim}
rm(<obj1>, <obj2>,...)
\end{verbatim}
For example, to remove the objects \verb+ACTpop+ and \verb+elasticband+ from the
workspace image before quitting, type

\begin{verbatim}
rm(ACTpop, elasticband)
q()
\end{verbatim}

\newpage
\section{Vectors in R}
Vectors may have mode logical, numeric or character.
 Examples of vectors are
\begin{verbatim}
> c(2,3,5,2,7,1)
[1] 2 3 5 2 7 1

> 3:10 # The numbers 3, 4,.., 10
[1] 3 4 5 6 7 8 9 10

> c(T, F, F, F, T, T, F)
[1] TRUE FALSE FALSE FALSE TRUE TRUE FALSE

> c("Canberra", "Sydney", "Canberra", "Sydney")
[1] "Canberra" "Sydney"   "Canberra" "Sydney"
\end{verbatim}
The first two vectors above are numeric, the third is logical
(i.e. a vector with elements of mode logical), and the fourth
is a string vector (i.e. a vector with elements of mode character).
\newpage
\subsection{Concatenation -- joining vector objects}
The \verb+c+ in \verb+c(2, 3, 5, 2, 7, 1)+ is an acronym for `concatenate'.
The meaning is: `Collect these numbers together to form a vector'.
We can concatenate two vectors. In the following, we form numeric
vectors \verb+x+ and \verb+y+, that we then concatenate to form
a vector \verb+z+:

\begin{verbatim}
> x <- c(2,3,5,2,7,1)
> x
[1] 2 3 5 2 7 1
> y <- c(10,15,12)
> y
[1] 10 15 12
> z <- c(x, y)
> z
[1] 2 3 5 2 7 1 10 15 12
\end{verbatim}
\newpage
\subsection{Subsets of vectors}
Note three common ways to extract subsets of vectors.

\begin{enumerate}
\item Specify the indices of the elements that are to be extracted,
e.g.,
\begin{verbatim}
> x <- c(3,11,8,15,12) # Assign to x the values 3, 11, 8, 15, 12
> x[c(2,4)]            # Extract the elements in positions 2 and 4
[1] 11 15
\end{verbatim}
\item Use negative subscripts to omit the elements in nominated subscript
positions:\footnote{Take care not to mix positive and negative subscripts.}
\begin{verbatim}
> x[-c(2,3)] # Remove the elements in positions 2 and 3
[1] 3 15 12
\end{verbatim}
\item Specify a vector of logical values. The elements that are
extracted are those for which the logical value is \verb+TRUE+.
Thus, suppose we want to extract values of \verb+x+ that are greater
than 10.
\begin{verbatim}
> x > 10
[1] FALSE TRUE FALSE TRUE TRUE
> x[x>10]
[1] 11 15 12
\end{verbatim}
\end{enumerate}
\newpage

\subsection{Missing values}
The missing value symbol is \verb!NA!.  As an example, we may set:
\begin{verbatim}
y <- c(1, NA, 3, 0, NA)
\end{verbatim}
Note that any arithmetic operation or relation that involves \verb+NA+
generates an \verb+NA+.  Specifically, be warned that
\verb+y[y==NA] <- 0+ leaves \verb+y+ unchanged.  The reason is that
all elements of \verb+y==NA+ evaluate to \verb+NA+.  This does not
identify an element of \verb+y+, and there is no assignment.  To
replace all \verb+NA+s by 0, use the function \verb+is.na()+, thus
\begin{verbatim}
> y[is.na(y)] <- 0
> y
[1] 1 0 3 0 0
\end{verbatim}
The functions \verb+mean()+, \verb+median()+, \verb+range()+, and a
number of other functions, take the argument \verb+na.rm=TRUE+; i.e.
remove \verb+NA+s, then proceed with the calculation.  By default,
these and related functions will fail when there are \verb+NA+s.  By
default, the \verb+table()+ function ignores \verb+NA+s.
\newpage

\subsubsection*{Setting the number of decimal places in output}

The \verb!options()! function can be
used to change to make a global change to the number of significant
digits that are printed.  For example:
\begin{verbatim}
> sqrt(10)
[1] 3.162278
options(digits=2)   # Change until further notice
> sqrt(10)
[1] 3.2
\end{verbatim}

\subsection{Factors}
A factor is a special type of vector, stored internally as a
numeric vector with values 1, 2, 3,\dots $k$. The value $k$ is the number
of levels.

Consider a survey that has data on 691 females and 692 males.
If the first 691 are females and the next 692 males, we can create
a vector of strings that that holds the values thus:
\begin{verbatim}
gender <- c(rep("female",691), rep("male",692))
\end{verbatim}
We can change this vector to a factor, by entering:
\begin{verbatim}
gender <- factor(gender)
\end{verbatim}
Internally, the factor \verb+gender+ is stored as 691 1's, followed
by 692 2's. It has stored with it a table that holds the information:

\smallskip
\begin{tabular}{|l|l|}
\hline
 1 & female \\
\hline
 2 & male \\
\hline
\end{tabular}
\smallskip

The
1 is translated into \verb!female! and the 2 into \verb!male!. The values
\verb!female! and \verb!male! are the levels of the factor. By default,
the levels are chosen to be in alphanumeric order, so that \verb!female!
precedes \verb!male!. Hence:
\begin{verbatim}
> levels(gender)
[1] "female" "male"
\end{verbatim}
Note that if \verb+gender+ had been an ordinary character vector, the outcome
of the above \verb+levels+ command would have been \verb+NULL+.
The order of the factor levels determines the order of appearance of
the levels in graphs and tables that use this information.  To cause
\verb!male! to come before \verb!female!, use
\begin{verbatim}
gender <- factor(gender, levels=c("male", "female"))
\end{verbatim}
This syntax is available both when the factor is first created, or
later to change the order in an existing factor.  Take care that the
level names are correctly spelled.  For example, specifying
\verb!"Male"! in place of \verb!"male"! in the \verb!levels!
argument will cause all values that were \verb!"male"!
to be coded as missing.

For an existing factor, a better way to change the order is:
\begin{verbatim}
levels(gender) <- c("male", "female")
\end{verbatim}

One advantage of factors is that the memory required for storage is 
usually less than for the corresponding character vector.

\section{Data Frames}\label{dataframes}
A data frame is a generalization of a matrix,
in which different columns may have different modes. All elements
of any column must, however, have the same mode, i.e.\ all numeric
or all factor, or all character, or all logical.

Included with our data sets is \verb+Cars93.summary+,
created from the \verb+Cars93+ data set in the Venables and
Ripley \textit{MASS} package, and included in our \textit{DAAG}
library.  In order to access it, we need to load it into the
workspace, thus:
\begin{verbatim}
> library{daag}          # load the DAAG library
> data(Cars93.summary)   # copy Cars93.summary into the workspace
> Cars93.summary
        Min.passengers Max.passengers No.of.cars abbrev
Compact              4              6         16      C
Large                6              6         11      L
Midsize              4              6         22      M
Small                4              5         21     Sm
Sporty               2              4         14     Sp
Van                  7              8          9      V
\end{verbatim}

The data frame has row labels (accessed using
\verb+row.names(Cars93.summary)+) \verb!Compact!, \verb!Large!,... The
column names (accessed using \verb+names(Cars93.summary)+) are
\verb+Min.passengers+ (i.e. the minimum number of passengers for cars
in this category), \verb+Max.passengers+, \verb+No.of.cars+, and
\verb+abbrev+. The first three columns have mode numeric, and the
fourth has mode character.  The column \verb+abbrev+ could equally
well be stored as a factor.

There are several ways to access the columns of a data frame.
Any of the following\footnote{Also allowed is \texttt{Cars93.summary[4]}.
This gives a data frame with the single column \texttt{abbrev}.}
will pick out the fourth column of the data frame \verb+Cars93.summary+
and store it in the vector \verb+type+.
\begin{verbatim}
type <- Cars93.summary$abbrev
type <- Cars93.summary[,4]
type <- Cars93.summary[,"abbrev"]
type <- Cars93.summary[[4]] # Take the object that is stored
                            # in the fourth list element.
\end{verbatim}
In each case, one can view the contents of the object \verb!type!:
\begin{verbatim}
> type
[1] C  L  M  Sm Sp V
Levels:  C L M Sm Sp V
\end{verbatim}

It is often convenient to use the \verb!attach()! function:
\begin{verbatim}
> attach(Cars93.summary)
  # R can now access the columns of Cars93.summary directly
> abbrev
[1] C  L  M  Sm Sp V
Levels:  C L M Sm Sp V
> detach("Cars93.summary")
  # Not strictly necessary, but tidiness is a good habit!
  # detach(Cars93.summary) is an acceptable alternative
\end{verbatim}
Detaching data frames that are no longer in use reduces the risk
of a clash of variable names, e.g., two different attached
data frames that have a column with the name \verb!abbrev!,
or an \verb!abbrev! both in the workspace and in an attached
data frame.

\newpage
In Windows versions, use of \verb+edit()+ allows access to a
spreadsheet-like display of a data frame or of a vector.  One can then
directly manipulate individual entries or perform data entry
operations as with a spreadsheet.  For example,
\begin{verbatim}
Cars93.summary <- edit(Cars93.summary)
\end{verbatim}
To close the spreadsheet, click on the \textbf{File} menu and then on
\textbf{Close}.

\newpage
\subsection{Variable names}
The \verb!names()! function can be used to determine variable names in
a data frame.  As an example, consider the New York air quality data
frame that is included with the base R package.  To determine the
variables in this data frame, type
\begin{verbatim}
data(airquality)
names(airquality)
[1] "Ozone"   "Solar.R" "Wind"    "Temp"    "Month"   "Day"
\end{verbatim}
The \verb!names()! function serves a second purpose.  To change the
name of the \verb!abbrev! variable (the fourth column) in the
\verb!Cars93.summary!  data frame to \verb!code!, type
\begin{verbatim}
names(Cars93.summary)[4] <- "code"
\end{verbatim}
If we want to change all of the names, we could do something like
\begin{verbatim}
names(Cars93.summary) <- c("minpass","maxpass","number","code")
\end{verbatim}

\newpage
\subsection{Applying a function to the columns of a dataframe}
\label{ssec:sapply}
The \verb!sapply()! function is a useful tool for calculating
statistics for each column of a data frame.  The first argument to
\verb!sapply()! is a data frame.  The second argument is the name of a
function that is to be applied to each column.  Consider the
\verb!women! data frame.
\begin{verbatim}
> data(women)
> women    # Display the data
   height weight
1      58    115
2      59    117
3      60    120
................
15     72    164
\end{verbatim}
In order to compute averages of each column, type:
\begin{verbatim}
> sapply(women, mean)    # Apply mean() to each of the columns
height weight
  65.0  136.7
\end{verbatim}
\newpage
\section{R packages}\label{sec:packages}
The default R distribution includes a number of packages in its
library.  Note in particular \textit{base}, \textit{eda}, \textit{ts}
(time series), \textit{boot} and \textit{MASS}. 

Installed packages, unless loaded automatically, must then be
\textit{load}ed prior to use.  The \textit{base} package is
automatically loaded at the beginning of the session. To load any
other installed package, use the \verb+library()+ function.  For
example,
\begin{verbatim}
library(mass)    # Loads the MASS library
\end{verbatim}
The \verb!require()! function has the same effect as the \verb!library()!
function.

\addtocounter{section}{1}

\section{R Graphics}\label{plotting}
The functions \verb+plot()+, \verb+points()+, \verb+lines()+, \verb+text()+,
\verb+mtext()+, \verb+axis()+, \verb+identify()+, etc. form a suite that 
plot graphs and
add features to the graph. To see some of the possibilities that
R offers, enter
\begin{verbatim}
demo(graphics)
\end{verbatim}
Press the Enter key to move to each new graph.
\newpage
\subsection{plot( ) and allied functions}\label{plotallies}
The basic command is
\begin{verbatim}
plot(y ~ x)
\end{verbatim}
or
\begin{verbatim}
plot(x, y)
\end{verbatim}
where \verb!x! and \verb!y! must be the same length.

Here are two examples.
\begin{verbatim}
attach(elasticband)     
plot(stretch, distance) # Alternative: plot(distance ~ stretch)
detach(elasticband)

attach(austpop)
plot(year, ACT, type="l") # Join the points ("l" = "line")
detach(austpop)
\end{verbatim}
\newpage
\subsubsection{Fine control -- parameter settings}
When it is necessary to change the default parameter settings, use the
\verb+par()+ function.  We have already used \verb!par(mfrow=c(m, n))!
to get an $m$ by $n$ layout of graphs on a page.
Here is another example:
\begin{verbatim}
 par(cex=1.25, mex=1.25)
\end{verbatim}
increases the text and plot symbol size 25\% above the default.
Adding \verb+mex=1.25+ makes room in the margin to accommodate
the increased text size.

One can store the existing settings, so that they can be restored
later. For this, specify, e.g.,
\begin{verbatim}
 oldpar <- par(cex=1.25, mex=1.25) # Use par(oldpar) to restore
\end{verbatim}
The size of the axis annotation can be controlled, independently of the setting of \verb!cex!,
by specifying a value for \verb!cex.axis!.  Similarly, \verb!cex.labels!
 may be used to control the size of the
axis labels.

Type in \verb+help(par)+ to get a list of all the parameter settings
that are available with \verb+par()+.

\newpage
\subsubsection*{Adding points, lines, text and axis annotation}
Use the \verb+points()+ function to add points to a plot. Use
the \verb+lines()+ function to add lines to a plot.\footnote{Actually
these functions are identical, differing only in the default
setting for the parameter \verb!type!. The default setting
for \verb!points()! is \verb!type = "p"!, and for \verb!lines()!
is \verb!type = "l"!.  Explicitly setting \verb!type = "p"!
causes either function to plot points, \verb!type = "l"! gives lines.}
The \verb+text()+ function
places text anywhere on the plot.

The function \verb+mtext(text, side, line, ..)+ adds text in the margin of
the current plot. The sides are numbered 1(x-axis), 2(y-axis),
3(top) and 4(right vertical axis).

The \verb+axis()+ function gives fine control over axis ticks
and labels.
\newpage
\subsubsection*{Labeling points -- an example}
Figure \ref{fig:primateplot} presents a scatterplot of the data from
the \verb+primates+ data frame.  Notice from the listing below that
the species names are held as row names, which we can access using the
function \verb+row.names()+:


\begin{figure}[H]
\begin{center}
\includegraphics{Pictures/g1-2.pdf}
\caption{Brain weight versus body weight, for
the primates data frame.}\label{fig:primateplot}
\end{center}
\end{figure}

\newpage
We begin with code that will give a crude version of Figure
\ref{fig:primateplot}.  We use the function \verb+text()+ to add text
labels to the points on a plot.

\begin{verbatim}
data(primates)    # The DAAG package must be loaded
attach(primates)  # Needed if primates is not already attached.
plot(Bodywt, Brainwt, xlim=c(5, 250))
# Specify xlim so that there is room for the labels
text(x=Bodywt, y=Brainwt, labels=row.names(primates), adj=0)
# adj=0 implies left adjusted text
detach(primates)
\end{verbatim}

The resulting graph would be adequate for identifying points,
but it is not a presentation quality graph. We now note the changes
that are needed to get Figure \ref{fig:primateplot}.

In Figure \ref{fig:primateplot} we use the \verb+xlab+ (x-axis) and
\verb+ylab+ (y-axis) parameters to specify meaningful axis titles. We
move the labeling to one side of the points by including appropriate
horizontal and vertical offsets. We multiply
\verb+chw <- par()$cxy[1]+ by 0.1 to get an horizontal offset that is
one tenth of a character width, and similarly for
\verb+chh <- par()$cxy[2]+ in a vertical direction.  We use
\verb+pch=16+ to make the plot character a heavy black dot. This helps
make the points stand out against the labeling.

\newpage
Here is the R code for Figure \ref{fig:primateplot}:

\begin{verbatim}
attach(primates)
plot(x=Bodywt, y=Brainwt, pch=16,
       xlab="Body weight (kg)", ylab="Brain weight (g)",
       xlim=c(5,300), ylim=c(0,1500))
chw <- par()$cxy[1]
chh <- par()$cxy[2]
text(x=Bodywt+chw, y=Brainwt+c(-.1,0,0,.1,0)*chh,
       labels=row.names(primates), adj=0)
detach(primates)
\end{verbatim}

\newpage
\subsubsection*{Rug plots}

The function \verb+rug()+ adds vertical bars, showing the distribution
of data values, along one or both of the $x$ and $y$ axes of an
existing plot.  Figure \ref{fig:milkrug}. has rugs on both the $x$ and
$y$ axes.  The code that produced Figure \ref{fig:milkrug} is:
\begin{verbatim}
data(milk)
xyrange <- range(milk)
plot(four ~ one, data = milk, xlim = xyrange, ylim = xyrange, pch = 16)
rug(milk$one)
rug(milk$four, side = 2)
abline(0, 1)
\end{verbatim}

\begin{figure}[H]
\begin{center}
\includegraphics{Pictures/g1-3.pdf}
\caption{Each of 17 panelists compared two milk samples
for sweetness. The $y$-ordinate is the assessment for four units
of additive, while the $x$-ordinate is the assessment for one unit
of additive. 
}\label{fig:milkrug}
\end{center}
\end{figure}

\addtocounter{subsection}{2}
\subsection{Identification and location on the figure region}
Two functions are available for this purpose.  Draw the graph first,
then call one or other of these functions:
\begin{itemize}
\item[$\circ$] \verb!identify()! labels
points.  One positions the cursor near the point that is to be
identified, and clicks the left mouse button.
\item[$\circ$] \verb!locator()! prints out
the co-ordinates of points.
\end{itemize}
The user positions the cursor at the location for which coordinates are
required, and clicks the left mouse button.  A click with the right
mouse button signifies that the identification or location task is
complete, unless the setting of the parameter \verb!n! is reached
first.  For \verb!identify()! the default setting of \verb!n! is the
number of data points, while for \verb!locator()! the default setting
is \verb!n! = 500.

As an example, try to identify two of the plotted points on
the \verb!primates! scatterplot:

\begin{verbatim}
attach(primates)
plot(Bodywt, Brainwt)
identify(Bodywt, Brainwt, n=2) # now click near 2 plotted points
detach(primates)
\end{verbatim}

\newpage
\subsection{Plotting mathematical symbols}
Both \verb+text()+ and \verb+mtext()+ allow replacement of the
text string by a mathematical expression. In \verb+plot()+,
either or both of \verb+xlab+ and \verb+ylab+ can be an algebraic
expression. Figure \ref{fig:mathexpplot} was produced with

\begin{verbatim}
p <- (0:100)/100
plot(p, sqrt(p*(1-p)), ylab=expression(sqrt(p(1-p))), type="l")
\end{verbatim}

\begin{figure}[H]
\begin{center}
\includegraphics{Pictures/g1-4.pdf}
\caption{The y-axis label is a mathematical expression.}\label{fig:mathexpplot}
\end{center}
\end{figure}

Type \verb+help(plotmath)+ to get details of available forms of mathematical
expression. The final plot from \verb+demo(graphics)+
shows some of the possibilities for plotting mathematical symbols.
\newpage
\subsection{Row by column layouts of plots}
There are several ways to do this.  Here, we will demonstrate two of
them.  

\subsubsection*{Multiple plots, each with its own margins}
As noted in earlier sections, the parameter \verb+mfrow+ can be used
to configure the graphics sheet so that subsequent plots appear row by
row, one after the other in a rectangular layout, on the one page. For
a column by column layout, use \verb+mfcol+. The following example
gives a plot that displays four different transformations of the
\verb!Animals!  data.

\begin{verbatim}
par(mfrow=c(2,2))
require(mass)       # Animals is in the MASS package
data(Animals)       # Needed if Animals is not already loaded
attach(Animals)
plot(body, brain)
plot(sqrt(body), sqrt(brain))
plot((body)^0.1, (brain)^0.1)
plot(log(body), log(brain))
detach("Animals")
par(mfrow=c(1,1))   # Restore to 1 figure per page
\end{verbatim}

\newpage
\subsubsection*{Multiple panels -- xyplot()}

The function \verb+xyplot()+ allows for an $x$ by $y$ layout of panels
in which the axis labeling appears in the outer margins. The
\textit{lattice} package substantially extends the abilities that are
available using the older function \verb+coplot()+. 

Try

\begin{verbatim}
> library(lattice)
> data(possum)      # DAAG must be loaded
> table(possum$Pop,  possum$sex)  # Graph reflects layout of this table
    Vic other
  f  24    19
  m  22    39
> xyplot(totlngth ~ age|sex*Pop, data=possum)
\end{verbatim}

The missing rows are due to missing values for the \verb+age+ variable.  Note
that the plot is constructed anyway, using the nonmissing data.

\begin{figure}[H]
\begin{center}
\includegraphics{Pictures/possum1.pdf}
\caption{Total length of possums versus age, for each
combination of population (the Australian state of Victoria or
elsewhere) and sex (female or male).} 
\end{center}
\end{figure}
\newpage
\subsection{Graphs -- additional notes}
\subsubsection*{The shape of the graph sheet}
It is often desirable to control the shape of the graph page.  For
example, we might want the individual plots to be rectangular rather
than square. The function \verb!x11()! sets up a graphics page on the
screen display. It takes arguments \verb!width! (in inches),
\verb!height!  (in inches) and \verb!pointsize! (in 1/72 of an inch).
The setting of \verb!pointsize! (default =12) determines character
heights.\footnote{It is the relative sizes of these parameters that
  matter for screen display or for incorporation into Word and similar
  programs.  Once pasted (from the clipboard) or imported into Word,
  graphs can be enlarged or shrunk by pointing at one corner, holding
  down the left mouse button, and pulling.}
\newpage
\subsubsection{Plot methods for objects other than vectors}
We have seen how to plot a numeric vector \verb+y+ against a numeric
vector \verb+x+.  The plot function is a generic function that also
has special methods for `plotting' various different classes of
object. For example, one can give a data frame as the argument to
\verb+plot+. Try

\begin{verbatim}
data(trees)    # Load data frame trees (base package)
plot(trees)    # Has the same effect as pairs(trees)
\end{verbatim}

The \verb!pairs()! function will be important when we come to discuss
multiple regression.  
\newpage
\subsubsection*{Good and bad graphs}

Draw graphs so that they are unlikely to mislead, make sure that they
focus the eye on features that are important, and avoid distracting
features.  In scatterplots, the intention is typically to draw
attention to the points. If there are not too many of them, drawing
them as heavy black dots or other symbols will focus attention on the
points, rather than on a fitted line or curve or on the axes.  If they
are numerous, dots are likely to overlap. It then makes sense to use
open symbols. Where there are many points that overlap, the ink will
be denser.  If there are many points, it can be helpful to plot
points in a shade of gray.\fc{For example, try:

\begin{BVerbatim}
plot(1:4, 1:4, pch=16, col=c("gray20","gray40","gray60","gray40"))
\end{BVerbatim}
}

Where the horizontal scale is continuous, patterns of change
that are important to identify should have an angle of slope in the
approximate range 20$^\circ$ to 70$^\circ$.  

There is a huge choice and range of colors.  Colors, or gray scales,
can often be used to useful effect to distinguish groupings in the
data.  Bear in mind that the eye has difficulty in focusing
simultaneously on widely separated colors that appear close together
on the same graph.






\end{document}
